\begin{definition}\label{def:cohomology_of_k-prsh}
    \uses{def:K_pre_sheaves}
    Let $\mathcal{F}^{\bullet}$ be complex of $\mathcal{K}$-presheaves then taking the cohomology defines a $\mathcal{K}$-presheaf denoted $H^{\bullet}\mathcal{F}^{\bullet}$.
\end{definition}

\begin{definition}\label{def:quasi_iso_of_complex_of_k-prsh}
    \uses{def:cohomology_of_k-prsh}
    A morphisme of complex of $\mathcal{K}$-presheave $\mathcal{F}^{\bullet}\to\mathcal{G}^{\bullet}$ is called quasi-isomorphism if it induces isomorphisms $H^i\mathcal{F}^{\bullet}\to H^i\mathcal{G}^{\bullet}$ for all $i$.
\end{definition}

\begin{definition}\label{def:homotopy_k_sheaf}
    \uses{def:K_pre_sheaves, def:total_complex}
    A complex of $\mathcal{K}$-presheaves $\mathcal{F}^{\bullet}$ is said to be a Homotopy-$\mathcal{K}$-sheave if the folowing conditions are satisfied:\begin{itemize}
        \item\begin{equation}\label{axiom:hKsh1}
            \mathcal{F}^{\bullet}(\varnothing) \text{ is an acyclic complex}
        \end{equation}
        \item For $K_1$ and $K_2$ two comapcts of $X$ the folowing complex is acyclic:\begin{equation}\label{axiom:hKsh2}
             [\mathcal{F}^{\bullet}(K_1\cup K_2)\to \mathcal{F}^{\bullet-1}(K_1)\bigoplus\mathcal{F}^{\bullet-1}(K_2)\to \mathcal{F}^{\bullet-2}(K_1\cap K_2) ]
        \end{equation}
        \item For any compact $K$ of $X$, the following natural morphism is a quasi-isomorphism \begin{equation}\label{axiom:hKsh3}
            \varinjlim\limits_{K\subset U\text{ open relatively compact}}\mathcal{F}^{\bullet}(\overline{U})\to \mathcal{F}^{\bullet}(K)
        \end{equation}
    \end{itemize}
\end{definition}

\begin{lemma}\label{lem:Mayer_Vietoris}
    \uses{lem:complex_total_of_three_is_acyclic}
    By using \ref{lem:complex_total_of_three_is_acyclic}, \eqref{axiom:hKsh3} give rise to a "Mayer-Vietoris" long exact sequence: \[\ldots \to H^k \mathcal{F}^{\bullet}(K_1\cup K_2)\to H^k \mathcal{F}^{\bullet}(K_1)\bigoplus H^k\mathcal{F}^{\bullet}(K_2)\to H^k \mathcal{F}^{\bullet}(K_1\cap K_2)\to \ldots\]
\end{lemma}

\begin{lemma}\label{lem:homotpy_k_sheaves_stable_by_extension}
    \uses{def:K_pre_sheaves,def:homotopy_k_sheaf}
    Let $\mathcal{F}^{\bullet}$ be a complex of $\mathcal{K}$-presheaves. If $\mathcal{F}^{\bullet}$ has a finite filtration whose associated graded is a Homotopy-$\mathcal{K}$-sheaf, then $\mathcal{F}^{\bullet}$ is a Homotopy-$\mathcal{K}$-sheaf.
\end{lemma}

\begin{proof}
    TODO
\end{proof}

\begin{lemma}\label{lem:first_non_zero_homology_of_homotopy_k_sheaf_is_k_sheaf}
    \uses{def:K_sheaves,def:homotopy_k_sheaf}
    If $\mathcal{F}^{\bullet}$ is a homotopy-$\mathcal{K}$-sheaf, and $H^{-1}\mathcal{F}^{\bullet}=0$ then $H^{0}\mathcal{F}^{\bullet}$ is a $\mathcal{K}$-sheaf
\end{lemma}

\begin{proof}
    \uses{lem:Mayer_Vietoris}
    \begin{itemize}
        \item $\mathcal{F}^{\bullet}(\varnothing)$ is acyclic then in particular it's cohomology in degre $0$ is $0$, then one gets \eqref{axiom:Ksh1}
        \item $H^{-1}\mathcal{F}^{\bullet}(K_1\cap K_2)=0$ then the first terms \ref{lem:Mayer_Vietoris} gives the exact sequence of \eqref{axiom:Ksh2}
        \item Let $K$ be a compact of $X$, the quasi-isomorphism of \eqref{axiom:Ksh3} gives (in particular) that $H^0\mathcal{F}^{\bullet}(K)=H^0(\varinjlim\mathcal{F}^{\bullet}(\bar{U}))$. To conclude one has to apply that the cohomology commute with inductive limit of a complex (Bouraki algèbre prop 1 X.28) and that in the category of presheaves of abélian groups, the limits are computed objectvise.
    \end{itemize}
\end{proof}



