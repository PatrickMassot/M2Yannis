\section{\v{C}ech cohomology of sheaves}

\begin{definition}\label{def:chech_of_sh}
    If $X$ is a topological space, and $\mathcal{F}$ a sheaf over $X$, then let $\check{H}^{\bullet}(X;\mathcal{F}))$ be the \v{C}ech cohomology of $X$ with coeficient in $\mathcal{F}$
\end{definition}

\begin{definition}\label{def:chech_of_compact_sup_of_sh}
    If $X$ is a topological space, $K$ a comapct subset of $X$ and $\mathcal{F}$ a sheaf over $X$, then let $\check{H}_K^{\bullet}(X;\mathcal{F}))$ be the \v{C}ech cohomology of $X$ with support in $K$ with coeficient in $\mathcal{F}$
\end{definition}

\begin{definition}\label{def:pullback_for_chech}
    \uses{def:chech_of_compact_sup_of_sh, def:chech_of_sh}
    Let $f:X\to Y$ be a continuous map between topological spaces, and $\mathcal{F}$ a sheaf over $X$. $f$ induces a natural map $\check{H}^{\bullet}(Y;f_*\mathcal{F})\to \check{H}^{\bullet}(X;\mathcal{F})$.\\

    Moreover if $f$ is proper, one gets a natural map $\check{H}_c^{\bullet}(Y;f_*\mathcal{F})\to \check{H}_c^{\bullet}(X;\mathcal{F})$

\end{definition}

\begin{lemma}\label{lem:chech_comm_with_exceptional}
    \uses{def:exceptional_pushforward_of_sheaf}
    Let $f:X\to Y$ be an inclusion of open subset, then there is a natural isomorphism $f_!:\check{H}_c^{\bullet}(X;\mathcal{F})\to \check{H}^{\bullet}(Y;f_!\mathcal{F})$
\end{lemma}

\begin{proof}

\end{proof}

\section{\v{C}ech cohomology of complex of $\mathcal{K}$-sheaves}

\begin{definition}\label{def:chech_of_compelx_of_k_prsh}
    Let $\mathcal{F}^{\bullet}$ be a complex of $\mathcal{K}$-presheaves on a compact space $X$ then we define the \v{C}ech cohomology $\check{H}(X;\mathcal{F}^{\bullet})$ by TODO
\end{definition}

\begin{remark}\label{def:chech_of_k_prsh}
    By using the inclusion of $\mathcal{K}$-presheaves into complexes of $\mathcal{K}$-presheave, one get's a definition of \v{C}ech cohomology for $\mathcal{K}$-presheave.
\end{remark}

\begin{lemma}\label{lem:chech_preserve_quasi_iso}
    \uses{def:quasi_iso_of_complex_of_k-prsh}
    If $\mathcal{F}^{\bullet}\to\mathcal{G}^{\bullet}$ is a quasi-isomorphism then the induced maps $\check{H}^i\mathcal{F}^{\bullet}\to \check{H}^i\mathcal{G}^{\bullet}$ are isomorphims.
\end{lemma}

\begin{proof}
    TODO
\end{proof}

\begin{proposition}\label{prop:homotopy_k_sheaf_compute_chech}
    \uses{def:homotopy_k_sheaf,def:chech_of_compelx_of_k_prsh}
    Let $\mathcal{F}^{\bullet}$ be a complex of $\mathcal{K}$-presheaves that verify \eqref{axiom:hKsh1} and \eqref{axiom:hKsh2} then the canonical map $H^{\bullet}\mathcal{F}^{\bullet}\to \check{H}^{\bullet}(X;\mathcal{F}^{\bullet})$ is an isomorphism.
\end{proposition}

\begin{proof}
    %use chech-cohomology and hshk (1 et 2)

\end{proof}

\section{\v{C}ech cohomology is determined by stalks}

\begin{lemma}\label{lem:chech_is_determined_by_stalks_1}
    \uses{def:cohomology_of_k-prsh, def:chech_of_compelx_of_k_prsh, def:K_stalk}
    Let $\mathcal{F}^{\bullet}$ be a complex of $\mathcal{K}$-presheaves that verify \eqref{axiom:Ksh3} and such that all the stalks are $0$ then $\check{H}^{\bullet}(X;\mathcal{F})=0$
    
\end{lemma}

\begin{proof}
    \uses{def:K_pre_sheaves}% axiom 3
\end{proof}

\begin{lemma}\label{lem:chech_is_determined_by_stalks_2}
    \uses{def:chech_of_compelx_of_k_prsh,def:homotopy_k_sheaf}
    Let $\mathcal{F}^{\bullet}$ be a complex of $\mathcal{K}$-presheaves that verify \eqref{axiom:hKsh3} and $H^i\mathcal{F}^{\bullet}=0$ for $i<<0$.
    
    Then if the stalks of $\mathcal{F}^{\bullet}$ are acyclics, $\check{H}^{\bullet}(X;\mathcal{F}^{\bullet})=0$

\end{lemma}

\begin{proof}
    \uses{lem:chech_is_determined_by_stalks_1,lem:chech_preserve_quasi_iso}
    TODO
\end{proof}


\begin{proposition}\label{prop:chech_is_determined_by_stalks_3}
    \uses{def:chech_of_compelx_of_k_prsh,def:homotopy_k_sheaf}
    Let $\mathcal{F}^{\bullet}$ and $\mathcal{G}^{\bullet}$ be complexes of $\mathcal{K}$-presheaves that verify \eqref{axiom:hKsh3} and $H^i\mathcal{F}^{\bullet}=H^i\mathcal{G}^{\bullet}=0$ for $i$ small enough.
    
    Then if a morphism $\mathcal{F}^{\bullet}\to \mathcal{G}^{\bullet}$ induces a quasi-isomorphism on stalks, $\check{H}^{\bullet}(X;\mathcal{F}^{\bullet})=\check{H}^{\bullet}(X;\mathcal{G}^{\bullet})$
\end{proposition}

\begin{proof}
    \uses{lem:chech_is_determined_by_stalks_2}
    %uses: 
    % équation A.4.9
\end{proof}