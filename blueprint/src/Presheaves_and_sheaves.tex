
Let $X$ be a locally compact Hausdorf space.

\section{Sheaves}


\begin{definition}\label{def:pre_sheaves}
    A presheave on $X$ is a contravariant functor from the category of open sets of $X$ to abélian groups.
\end{definition}

\begin{definition}\label{def:stalk}
    \uses{def:pre_sheaves}
    If $\mathcal{F}$ is a presheaf on $X$ and $p\in X$ then the stalk of $\mathcal{F}$ at $p$ is the abelian group $\mathcal{F}_p:=\varinjlim\limits_{p\in U\text{ open}}\mathcal{F}(U)$.
\end{definition}

\begin{definition}\label{def:sheaves}
    \uses{def:pre_sheaves}
    If $\mathcal{F}$ is a presheaf on $X$, it is said to be a sheaf if for any $U\subset X$ open and any covering family of $U$ $(U_a)_{a\in A}$ one has the exact sequence:
    \begin{equation}\label{axiom:Sh}
        0\to \mathcal{F}(U)\to \prod\limits_{a\in A}\mathcal{F}(U_a)\to \prod\limits_{a,b\in A}\mathcal{F}(U_a\cap U_b)
    \end{equation}
\end{definition}

\section{$\mathcal{K}$-sheaves}

\begin{definition}\label{def:K_pre_sheaves}
    A $\mathcal{K}$-presheave on $X$ is a contravariant functor from the category of compact sets of $X$ to abélian groups.
\end{definition}

\begin{definition}\label{def:K_stalk}
    \uses{def:K_pre_sheaves}
    %\lean{M2.K-stalks.}
    If $\mathcal{F}$ is a $\mathcal{K}$-presheaf on $X$ and $p\in X$ then the stalk of $\mathcal{F}$ at $p$ is the abelian group $\mathcal{F}_p:=\varinjlim\limits_{p\in K\text{ compact}}\mathcal{F}(K)=\mathcal{F}(\{p\})$.  
\end{definition}

\begin{definition}\label{def:K_sheaves}
    %\lean{M2.Ksheaves.Ksheaf}
    \uses{def:K_pre_sheaves}
    If $\mathcal{F}$ is a $\mathcal{K}$-presheaf on $X$, it is said to be a $\mathcal{K}$-sheaf if the folowing conditions are satisfied:\begin{itemize}
    \item\begin{equation}\label{axiom:Ksh1}
        \mathcal{F}(\varnothing)=0
    \end{equation}
    \item For $K_1$ and $K_2$ two comapcts of $X$ the folowing sequence is exact:\begin{equation}\label{axiom:Ksh2}
         0\to\mathcal{F}(K_1\cup K_2)\to \mathcal{F}(K_1)\bigoplus\mathcal{F}(K_2)\to \mathcal{F}(K_1\cap K_2) 
    \end{equation}
    \item For any compact $K$ of $X$, the following natural morphism is an isomorphism\begin{equation}\label{axiom:Ksh3}
        \varinjlim\limits_{K\subset U\text{ open relatively compact}}\mathcal{F}(\overline{U})\to \mathcal{F}(K)
    \end{equation}
\end{itemize}
\end{definition}

\begin{remark}
    \eqref{axiom:Ksh3} is well defined because if $K$ is a compact subset of $X$, then for $x\in K$ let $U_x$ be an open neighborhood relatively compact (wich exists by local compactness), the family $(U_x)\ind{x}{K}$ covers $K$ then one can extract a finite cover of it : $U_1,\ldots U_n$ and then $\cup_{i=1}^n U_i$ is an open neighborhood, and a finite union of relatively comapct, then it's relatively compact. ($\overline{\cup_{i=1}^n U_i}=\cup_{i=1}^n \overline{U_i}$)
\end{remark}


\section{Technical lemmas}

\begin{lemma}\label{lem:cofinal_syst_of_inter_compact}
    If $K_1,\ldots K_n$ are comapcts of $X$ then $\{U_1\cap\ldots\cap U_n\}_{U_i\supset K_i\text{ open in }X}$ is a cofinal system of neighborhoods of $K_1\cap \ldots \cap K_n$.
\end{lemma}

\begin{proof}
    It's the theorem IsCompact.nhdsSet-inter-eq  in the File Mathlib/Topology/Separation.lean and the use of Filter.HasBasis.inf in the file  Mathlib.Order.Filter.Bases

   %Let $U_i$ be a relatively comapct open neighborhood of $K_i$ and $U=\cup_{i=1}^nU_i$. Then $\overline{U}$ is compact \\

   %If $n=2$, let $V$ be a neighborhood of $K_1\cap K_2$. By considering $U\cap V$, one ca assume that $V\subset U$.

   %If the result is true for $n-1$, 
\end{proof}

\begin{lemma}\label{lem:equiv_of_adj}
    If $\mathcal{C}$ and $\mathcal{D}$ are two categories, $F:\mathcal{C}\to \mathcal{D}$ and $G:\mathcal{D}\to \mathcal{C}$ two functors such that $(F,G)$ is an adjoint pair. Then for $(F,G)$ to be an equivalence of category, it's enough to have that thes canonical naturals transformations $\text{id}_{\mathcal{D}}\Rightarrow F\circ G$ and $G\circ F\Rightarrow \text{id}_{\mathcal{D}}$ are isomorphisms.
\end{lemma}

\begin{proof}
    CategoryTheory.Adjunction.toEquivalence in mathlib
\end{proof}

%\begin{lemma}\label{lem:a_nommer}
%    \uses{def:K_pre_sheaves nh}
%    If $(K_a)_{a\in A}$ is a filtered directed system of comapcts substes of $X$, and $\mathcal{F}$ a $\mathcal{K}$-presheaf satisfying\eqref{axiom:Ksh3}, then \[\varinjlim\limits_{a\in A}\mathcal{F}(K_a)\to \mathcal{F}(\bigcap\limits_{a\in A}K_a)\] is an isomorphism.
%\end{lemma}
%\begin{proof}
%    TODO
%\end{proof}

\section{Equivalence of category}

\begin{definition}\label{def:adj_kprshv_and_prshv}
    \uses{def:K_pre_sheaves,def:pre_sheaves}
    \begin{itemize}
        \item If $\mathcal{F}$ is a presheaf then let $\alpha^*\mathcal{F}$ be the $\mathcal{K}$-presheaf:\[K\mapsto \varinjlim\limits_{K\subset U \text{ open}}\mathcal{F}(U)\]
        \item If $\mathcal{G}$ is a $\mathcal{K}$-presheaf then let $\alpha_*\mathcal{G}$ be the presheaf:\[ U\mapsto \varprojlim\limits_{U\supset K \text{ compact}}\mathcal{G}(K)\]
    \end{itemize}
\end{definition}

\begin{proposition}\label{pro:adj_kprshv_and_prshv}
    \uses{def:adj_kprshv_and_prshv}
    The pair $(\alpha^*,\alpha_*)$ is an adjonit pair.
\end{proposition}

\begin{proof}
    \begin{itemize}
        \item Let $\tau$ be an element of $\hom(\alpha^*\mathcal{F},\mathcal{G})$. It's the data of morphism $\tau_K$ for $K$ a compact of $X$ such that for any $K$ and $K'$ compacts 
        % https://q.uiver.app/#q=WzAsNCxbMCwwLCJcXHZhcmluamxpbVxcbGltaXRzX3tLXFxzdWJzZXQgVX1cXG1hdGhjYWx7Rn0oVSkiXSxbMCwyLCJcXHZhcmluamxpbVxcbGltaXRzX3tLJ1xcc3Vic2V0IFV9XFxtYXRoY2Fse0Z9KFUpIl0sWzIsMCwiXFxtYXRoY2Fse0d9KEspIl0sWzIsMiwiXFxtYXRoY2Fse0d9KEsnKSJdLFswLDFdLFswLDIsIlxcdGF1X0siXSxbMSwzLCJcXHRhdV97Syd9Il0sWzIsM11d
\begin{equation}\label{nat_trans_afg}\begin{tikzcd}
	{\varinjlim\limits_{K\subset U}\mathcal{F}(U)} && {\mathcal{G}(K)} \\
	\\
	{\varinjlim\limits_{K'\subset U}\mathcal{F}(U)} && {\mathcal{G}(K')}
	\arrow["{\tau_K}", from=1-1, to=1-3]
	\arrow[from=1-1, to=3-1]
	\arrow[from=1-3, to=3-3]
	\arrow["{\tau_{K'}}", from=3-1, to=3-3]
\end{tikzcd}\end{equation} is a commutative square. Then for any $U$ and $V$ opens,  by composing with the commutative square % https://q.uiver.app/#q=WzAsNCxbMiwwLCJcXHZhcmluamxpbVxcbGltaXRzX3tLXFxzdWJzZXQgVX1cXG1hdGhjYWx7Rn0oVSkiXSxbMiwyLCJcXHZhcmluamxpbVxcbGltaXRzX3tLJ1xcc3Vic2V0IFV9XFxtYXRoY2Fse0Z9KFUpIl0sWzAsMCwiXFxtYXRoY2Fse0Z9KFUpIl0sWzAsMiwiXFxtYXRoY2Fse0Z9KFYpIl0sWzAsMV0sWzIsMF0sWzIsM10sWzMsMV1d
$$\begin{tikzcd}
	{\mathcal{F}(U)} && {\varinjlim\limits_{K\subset U}\mathcal{F}(U)} \\
	\\
	{\mathcal{F}(V)} && {\varinjlim\limits_{K'\subset U}\mathcal{F}(U)}
	\arrow[from=1-1, to=1-3]
	\arrow[from=1-1, to=3-1]
	\arrow[from=1-3, to=3-3]
	\arrow[from=3-1, to=3-3]
\end{tikzcd}$$ one get's a commutative square :% https://q.uiver.app/#q=WzAsNCxbMCwwLCJcXG1hdGhjYWx7Rn0oVSkiXSxbMCwyLCJcXG1hdGhjYWx7Rn0oVikiXSxbMiwwLCJcXG1hdGhjYWx7R30oSykiXSxbMiwyLCJcXG1hdGhjYWx7R30oSycpIl0sWzAsMV0sWzIsM10sWzEsM10sWzAsMl1d
\begin{equation}\label{data_adj}\begin{tikzcd}
	{\mathcal{F}(U)} && {\mathcal{G}(K)} \\
	\\
	{\mathcal{F}(V)} && {\mathcal{G}(K')}
	\arrow[from=1-1, to=1-3]
	\arrow[from=1-1, to=3-1]
	\arrow[from=1-3, to=3-3]
	\arrow[from=3-1, to=3-3]
\end{tikzcd}\end{equation}. Conversely such data give rise (by taking the limit over $U$ and $V$) to a commutative square such as in \eqref{nat_trans_afg}

    \item On the other hand if one takes the limit over $K$ and $K'$ one get's a commutative square 
    % https://q.uiver.app/#q=WzAsNCxbMCwwLCJcXG1hdGhjYWx7Rn0oVSkiXSxbMCwyLCJcXG1hdGhjYWx7Rn0oVikiXSxbMiwwLCJcXHZhcnByb2psaW1cXGxpbWl0c197S1xcc3Vic2V0IFV9XFxtYXRoY2Fse0d9KEspIl0sWzIsMiwiXFx2YXJwcm9qbGltXFxsaW1pdHNfe0tcXHN1YnNldCBWfVxcbWF0aGNhbHtHfShLKSJdLFswLDFdLFsyLDNdLFsxLDNdLFswLDJdXQ==
$$\begin{tikzcd}
	{\mathcal{F}(U)} && {\varprojlim\limits_{K\subset U}\mathcal{G}(K)} \\
	\\
	{\mathcal{F}(V)} && {\varprojlim\limits_{K\subset V}\mathcal{G}(K)}
	\arrow[from=1-1, to=1-3]
	\arrow[from=1-1, to=3-1]
	\arrow[from=1-3, to=3-3]
	\arrow[from=3-1, to=3-3]
\end{tikzcd}$$ (that allow to recover the previous one in the same as before) wich is the data of an element of $\hom(\mathcal{F},\alpha_*\mathcal{G})$.
\end{itemize}
    Then the elements of $\hom(\alpha^*\mathcal{F},\mathcal{G})$ and $\hom(\mathcal{F},\alpha_*\mathcal{G})$ are both obtained by a natural construction (in $\mathcal{F}$ and $\mathcal{G}$) aplied to \eqref{data_adj}.
\end{proof}

\begin{lemma}\label{lem:adj_kshv_and_shv}
    \uses{def:adj_kprshv_and_prshv}
    \begin{itemize}
        \item $\alpha^*$ send sheaves to $\mathcal{K}$-sheaves
        \item $\alpha_*$ send $\mathcal{K}$-sheaves to sheaves
        \item The reistrictions obtained still form an adjoint pair between shaeves and $\mathcal{K}$-sheaves.
    \end{itemize}
\end{lemma}

\begin{proof}
    \uses{def:K_sheaves, def:sheaves, lem:cofinal_syst_of_inter_compact}
    \begin{itemize}
        \item Let $\mathcal{F}$ be a sheaf. The condition $\varnothing \subset U$ is always satisfied and $\varnothing$ is minimal among open subset for the inclusion then $(\alpha^*)(\mathcal{F})(\varnothing)=\mathcal{F}(\varnothing)$. One can apply the sheaf condition to the empty family and obtain the exact sequence $0\to \mathcal{F}(\varnothing)\to \Pi_{\varnothing}=0$, and then \eqref{axiom:Ksh1}.\\

        Let $K_1,K_2$ be two of compacts of $X$, let $U_1,U_2$ be a two opens such that $K_i\subset U_i$ for all $i$. Then the sheaf condition gives an exact sequence (because for abelian groups the product is the direct sum) $0\to \mathcal{F}(U_1\cup U_2)\to \mathcal{F}(U_1)\bigoplus\mathcal{F}(U_2)\to \mathcal{F}(U_1\cap U_2)$. The injective limits are exacts then taking the limits over those opens gives an exact sequence: 
        \begin{equation}\label{lim_of_sheaf_cond}
            0\to \varinjlim\limits_{K_i\subset U_i}\mathcal{F}(U_1\cup U_2)\to \varinjlim\limits_{K_i\subset U_i}\mathcal{F}(U_1)\times\mathcal{F}(U_2)\to \varinjlim\limits_{K_i\subset U_i}\mathcal{F}(U_1\cap U_2)
        \end{equation}

        An open $U$ contains $K_1\cup K_2$ if and only if it's of the form $U_1\cup U_2$ with $K_i\subset U_i$ (one can take $U_1=U_2=U$ for the direct implication), then by definition $\varinjlim\limits_{K_i\subset U_i}\mathcal{F}(U_1\cup U_2)=\alpha^*\mathcal{F}(K_1\cup K_2)$.\\
        
        The injective limit commute with the direct sum (as it is a coproduct), then: \[\varinjlim\limits_{K_i\subset U_i}\mathcal{F}(U_1)\times\mathcal{F}(U_2)=(\varinjlim\limits_{K_i\subset U_i}\mathcal{F}(U_1))\bigoplus(\varinjlim\limits_{K_i\subset U_i}\mathcal{F}(U_2))=\alpha^*\mathcal{F}(K_1)\bigoplus\alpha^*\mathcal{F}(K_2)\].\\
        
        By the lemma \ref{lem:cofinal_syst_of_inter_compact} the limit $\varinjlim\limits_{K_i\subset U_i}\mathcal{F}(U_1\cap U_2)$ compute the same thing as $\varinjlim\limits_{K_1\cap K_2\subset U}\mathcal{F}(U)=\alpha^*\mathcal{F}(K_1\cap K_2)$.\\

        Then the exact sequence \eqref{lim_of_sheaf_cond} is in fact \eqref{axiom:Ksh2}.\\ \\

        Let $K$ be a compact, $U$ a relatively comapct open such that $K\subset U$ and $V$ an open suche that $\overline{U}\subset V$ then $K\subset V$. Conversely if $V$ is an open containing $K$, then $K$ is a comapct of $V$ (locally compact as $X$ is) and then admits an open neighborhood $U$ relatively compact (in $V$).Thus (because the two limits are over the same set) one has the equality \[\varinjlim\limits_{K\subset U \text{open relatively compact}}\varinjlim\limits_{\overline{U}\subset V\text{ open}}\mathcal{F}(V)=\varinjlim\limits_{K\subset U\ \text{open}}\mathcal{F}(V)\]. Wich rewrite by definition as $\varinjlim\limits_{K\subset U\ \text{open relatively compact}}\alpha^*\mathcal{F}(\overline{U})=\alpha^*\mathcal{F}(V)$ i.e. \eqref{axiom:Ksh3}.\\

        Then $\alpha^*\mathcal{F}$ is a $\Kc$-sheaf. 

        \item Let $\mathcal{G}$ be a $\Kc$-sheaf. Let $K_1,\ldots K_n$ be a family of compacts subsets let's show that the sequence $0\to \mathcal{G}(\bigcup\limits_{i=1}^n K_i)\to \prod\limits_{i=1}^n\mathcal{G}(K_i)\to \prod\limits_{i,j=1}^n\mathcal{G}(K_i\cap K_j)$.

        If the family is empty, then the sequence is $0\to \mathcal{G}\to 0\to 0$ wich is exact beause of \eqref{axiom:Ksh1}.

        If $n=1$ then the sequence is $0\to \mathcal{G}(K_1)\to \mathcal{G}(K_2)\to 0$ wich is exact beause of \eqref{axiom:Ksh1}.

        $\mathcal{G}$ is a $\Kc$-sheaf, then (by \eqref{axiom:Ksh2}) the map $\mathcal{G}(K\cup K')\to \mathcal{G}(K)\bigoplus\mathcal{G}(K')$ is injective, then (the direct products of two abélian groups is their product), then bi straightforward induction the map is injective  $\mathcal{G}(\bigcup\limits_{i=1}^nK_i)\to \prod\limits_{i=1}^n\mathcal{G}(K_i)$.

        Let's show the exactness of the other term in the sequence by induction, the base case of $n=2$ is given by \eqref{axiom:Ksh2}. Let's assume that it's exact for $n\in \mathbbm{N}$ fixed. Let $K_1,\ldots K_n, K_{n+1}$ be compacts subset and $(f_1,\ldots,f_{n+1})$ be an element of the kernel of  $\prod\limits_{i=1}^{n+1}\mathcal{G}(K_i)\to \prod\limits_{i,j=1}^{n+1}\mathcal{G}(K_i\cap K_j)$. Then $(f_1,\ldots,f_n)$ is in the kernel of $\prod\limits_{i=1}^n\mathcal{G}(K_i)\to \prod\limits_{i,j=1}^n\mathcal{G}(K_i\cap K_j)$ so by induction hypothesis, it's of the form $(f|_{K_1},\ldots,f|_{K_n})$ for $f\in \mathcal{G}(K:=\bigcup\limits_{i=1}^nK_i)$. 
        On the other hand, by the compatibility of reistriction $f|_{K\cap K_{n+1}}=f|_{K_1}|_{K\cap K_{n+1}}=f_1|_{K\cap K_{n+1}}=f_1|_{K_1\cap K_{n+1}}|_{K\cap K_{n+1}}$ so $f|_{K\cap K_n+1}-f_{n+1}|_{K\cap K_n+1}=f_1|_{K_1\cap K_{n+1}}|_{K\cap K_{n+1}}-f_{n+1}|_{K_1\cap K_{n+1}}|_{K\cap K_{n+1}}=0$ by hypothesis. Then by the exactness of \eqref{axiom:Ksh2} $f$ and $f_{n+1}$ are of the form $g|_K$ and $g|_{K_{n+1}}$, with $g\in \mathcal{G}(\bigcup\limits_{i=1}^{n+1} K_i)$.
        
        Then $(f_1,\ldots,f_{n+1})$ is of the form $(g|_{K_1},\ldots g|_{K_{n+1}})$. Wicch conclude the exactness prof because the inclusion of the image into the kernel is starightforward by definition of the map.

        Let $(U_a)\ind{a}{A}$ be a family of opens of $X$, one can consider the collections of family of compacts $(K_a)\ind{a}{A}$  such that $\forall a\in A /\ K_a\subset U_a$ and only  a finite number of them are not empty (by the \eqref{axiom:Ksh1}adding enpty compacts in the family ad a product with zero in the exact sequence, wich does not changes the sequence) and take the projective limit of the previous exact sequence over it.

        The sequence remains exact because projective limits are left exacts: \[0\to\varprojlim\mathcal{G}(\bigcup\limits_{a\in A}K_a)\to \varprojlim\prod\limits_{a\in A}\mathcal{G}(K_a)\to \varprojlim\prod\limits_{a,b\in A}\mathcal{G}(K_a\cap K_b)\]. The projective limits commmute with products, then the sequence is 

        \[0\to\varprojlim\mathcal{G}(\bigcup\limits_{a\in A}K_a)\to \prod\limits_{a\in A}\varprojlim\mathcal{G}(K_a)\to\prod\limits_{a,b\in A}\varprojlim\mathcal{G}(K_a\cap K_b)\]

        By definition (because it's for a fixed $a$ and does not depend of the family for other indexes) $\varprojlim\mathcal{G}(K_a)=\alpha_*\mathcal{G}(U_a)$ and $\varprojlim\mathcal{G}(K_a\cap K_b)=\alpha_*\mathcal{G}(U_a\cap U_b)$. Any compact included in $\bigcup\limits\ind{a}{A}U_a$ is included in a finite number of the opens then $\varprojlim\mathcal{G}(\bigcup\limits_{a\in A}K_a)$ compute $\alpha_*\mathcal{G}(\bigcup\limits\ind{a}{A}U_a)$, then one get's the sheaf condition for $\alpha_*\mathcal{G}$.

        \item A morphisme between two ($\mathcal{K}$-)sheaves is by definition is by definition a morphisme between the two underling ($\mathcal{K}$-)presheaves then, the natural equality $\hom_{\text{Sh}}(\alpha^*\mathcal{F},\mathcal{G})=\hom_{\text{Sh}}(\mathcal{F},\alpha_*\mathcal{G})$ is a consequence of \ref{pro:adj_kprshv_and_prshv}
    \end{itemize}
\end{proof}

%\begin{lemma}\label{lem:stalk_preserve}
%    \uses{def:adj_kprshv_and_prshv,def:stalk,def:K_stalk}
%    The functors $\alpha^*$ and $\alpha_*$ preserve the stalk.
%\end{lemma}

%\begin{proof}
%    \begin{itemize}
%        \item Let $\mathcal{F}$ be a sheaf. Then the stalk of $\alpha^*\mathcal{F}$ at $p\in X$ is by definition $\alpha^*\mathcal{F}(\{p\})=\varinjlim\limits_{\{p\}\subset U \text{ open}}\mathcal{F}(U)$ wich is by definition the stalk of $\mathcal{F}$ at $p$.
%        \item Let $\mathcal{G}$ be a $\Kc$-sheaf. Then 
%    \end{itemize}

    %the proof of this one should not be more complicated than the general case
%\end{proof}


\begin{lemma}\label{lem:kshv_equiv_shv}
    \uses{def:K_sheaves, def:sheaves, lem:adj_kshv_and_shv}
    The previous adjoint pair give rise to an equivalence of category between shaeves and $\mathcal{K}$-sheaves
\end{lemma}

\begin{proof}
    \uses{def:K_sheaves, def:sheaves, lem:adj_kshv_and_shv, lem:equiv_of_adj,def:stalk,def:K_stalk}
    By using \ref{lem:equiv_of_adj}, it's enough to show that for any sheaf $\mathcal{F}$ and $\Kc$-sheaf $\mathcal{G}$, the natural maps $\mathcal{F}\to \alpha_*\alpha^*\mathcal{U}$ and $\alpha^*\alpha_*\mathcal{G}\to \mathcal{G}$ are isomorphism. The fact of being a natural isomorphism can be checked locally.\begin{itemize}

        \item Let $K$ be a comapct of $X$. One has to check that $\varinjlim\limits_{K\subset U \text{ open}}\varprojlim\limits_{U\supset K' \text{ compact}}\mathcal{G}(K')\to \mathcal{G}(K)$ is an isomorphism.\\

        Let $U$ be an open relatively compact that contain $K$, for any $K'\subset U$ comapct,$\mathcal{G}$ define compatible maps $\mathcal{G}(\overline{U})\to \mathcal{G}(K')$, then by the universal proprety of the projective limit one get's a map $\mathcal{G}(\overline{U})\to \varprojlim\limits_{U\supset K'}\mathcal{G}(K')$ such that the map $\mathcal{G}(\overline{U})\to \varprojlim\limits_{U\supset K'}\mathcal{G}(K')\to \mathcal{G}(K)$ is $\mathcal{G}(\overline{U})\to \mathcal{G}(K)$. Then by taking the inductive limit over $U$, one get's ($\mathcal{G}(K)$ does not depend on $U$) that the canonical morphism $\varinjlim\limits_{K\subset U}\mathcal{G}(\overline{U})\to \mathcal{G}(K)$ factors that way: $\varinjlim\limits_{K\subset U}\mathcal{G}(\overline{U})\to \varinjlim\limits_{K\subset U}\varprojlim\limits_{U\supset K'}\mathcal{G}(K')\to \mathcal{G}(K)$. Then by $\eqref{axiom:Ksh3}$ it's enough to show that the map \[\varinjlim\limits_{K\subset U \text{open relatively compact}}\mathcal{G}(\overline{U})\to \varinjlim\limits_{K\subset U \text{open relatively compact}}\varprojlim\limits_{U\supset K' \text{compact}}\mathcal{G}(K')\] is an isomorphism.\\
        
        Let's build the map in the other direcction. By using the universal property of the inductive limit, one needs for any open relatively comapct $U$ that contains $K$ to build maps (compatibles with inclusion of opens) \[\varprojlim\limits_{U\supset K' \text{ compact}}\mathcal{G}(K')\to \varinjlim\limits_{K\subset U \text{ open relatively comapct}}\mathcal{G}(\overline{U})\]

        $K$ is a compact of $U$, then let $V$ be an open subset (of $U$ then of $X$) such that $\overline{V}\subset U$. $\overline{V}$ is a compact in $U$ then there is a canonical projection $\varprojlim\limits_{U\supset K' \text{ compact}}\mathcal{G}(K')\to \mathcal{G}(\overline{V})$ and $V$ is an open relatively comapct that contains $K$ then there is a canonical inclusion $\mathcal{G}(\overline{V})\to \varinjlim\limits_{K\subset U \text{ open relatively comapct}}\mathcal{G}(\overline{U})$, the composition of the two maps gives the desired morphism.\\

        One has to chack that it does not depend of the choice of $V$. If $V'$ is an other choice, then $V\cup V'$ is an open subset of $U$ that contains $K$ and $\overline{V\cup V'}=\overline{V}\cup\overline{V'}\subset U$. $V\cap V'$ is an open subset of $U$ that contains $K$ and $\overline{V\cap V'}\subset \overline{V}\cap\overline{V'}\subset U$, then all the triangle of the following diagram are commutative (and thus the two morphism are equal):

        % https://q.uiver.app/#q=WzAsNixbMCwxLCJcXHZhcnByb2psaW1cXGxpbWl0c197VVxcc3Vwc2V0IEsnfVxcbWF0aGNhbHtHfShLJykiXSxbMSwxLCJcXG1hdGhjYWx7R30oXFxvdmVybGluZXtWXFxjdXAgVid9KSJdLFsyLDAsIlxcbWF0aGNhbHtHfShcXG92ZXJsaW5le1YnfSkiXSxbMiwyLCJcXG1hdGhjYWx7R30oXFxvdmVybGluZXtWJ30pIl0sWzMsMSwiXFxtYXRoY2Fse0d9KFxcb3ZlcmxpbmV7VlxcY2FwIFYnfSkiXSxbNCwxLCJcXHZhcmluamxpbVxcbGltaXRzX3tLXFxzdWJzZXQgVX1cXG1hdGhjYWx7R30oXFxiYXJ7VX0pIl0sWzAsMV0sWzEsMl0sWzEsM10sWzIsNF0sWzMsNF0sWzQsNV0sWzIsNV0sWzMsNV0sWzAsMl0sWzAsM11d
\[\begin{tikzcd}
	&& {\mathcal{G}(\overline{V'})} \\
	{\varprojlim\limits_{U\supset K'}\mathcal{G}(K')} & {\mathcal{G}(\overline{V\cup V'})} && {\mathcal{G}(\overline{V\cap V'})} & {\varinjlim\limits_{K\subset U}\mathcal{G}(\overline{U})} \\
	&& {\mathcal{G}(\overline{V'})}
	\arrow[from=1-3, to=2-4]
	\arrow[from=1-3, to=2-5]
	\arrow[from=2-1, to=1-3]
	\arrow[from=2-1, to=2-2]
	\arrow[from=2-1, to=3-3]
	\arrow[from=2-2, to=1-3]
	\arrow[from=2-2, to=3-3]
	\arrow[from=2-4, to=2-5]
	\arrow[from=3-3, to=2-4]
	\arrow[from=3-3, to=2-5]
\end{tikzcd}\]

    If $W$ is an open relatively compact that contains $U$ then (by universal property of the projective limit) there is a commutative triangle :

    % https://q.uiver.app/#q=WzAsMyxbMCwwLCJcXHZhcnByb2psaW1cXGxpbWl0c197VVxcc3Vwc2V0IEsnfVxcbWF0aGNhbHtHfShLJykiXSxbMiwwLCJcXG1hdGhjYWx7R30oXFxvdmVybGluZXtWfSkiXSxbMCwxLCJcXHZhcnByb2psaW1cXGxpbWl0c197Vlxcc3Vwc2V0IEsnfVxcbWF0aGNhbHtHfShLJykiXSxbMCwxXSxbMiwwXSxbMiwxXV0=
\[\begin{tikzcd}
	{\varprojlim\limits_{U\supset K'}\mathcal{G}(K')} && {\mathcal{G}(\overline{V})} \\
	{\varprojlim\limits_{W\supset K'}\mathcal{G}(K')}
	\arrow[from=1-1, to=1-3]
	\arrow[from=2-1, to=1-1]
	\arrow[from=2-1, to=1-3]
\end{tikzcd}\]
    Then the triangle is also commutative:% https://q.uiver.app/#q=WzAsMyxbMCwwLCJcXHZhcnByb2psaW1cXGxpbWl0c197VVxcc3Vwc2V0IEsnfVxcbWF0aGNhbHtHfShLJykiXSxbMiwwLCJcXHZhcmluamxpbVxcbGltaXRzX3tLXFxzdWJzZXQgVX1cXG1hdGhjYWx7R30oXFxvdmVybGluZXtVfSkiXSxbMCwxLCJcXHZhcnByb2psaW1cXGxpbWl0c197V1xcc3Vwc2V0IEsnfVxcbWF0aGNhbHtHfShLJykiXSxbMCwxXSxbMiwwXSxbMiwxXV0=
    \[\begin{tikzcd}
        {\varprojlim\limits_{U\supset K'}\mathcal{G}(K')} && {\varinjlim\limits_{K\subset U}\mathcal{G}(\overline{U})} \\
        {\varprojlim\limits_{W\supset K'}\mathcal{G}(K')}
        \arrow[from=1-1, to=1-3]
        \arrow[from=2-1, to=1-1]
        \arrow[from=2-1, to=1-3]
    \end{tikzcd}\]

    That concludes the compatibility condition.\\

    Let's now check that the two maps are the inverse of one another. If $U$ and $V$ are opens relatively compacts of $X$ such that $K\subset V\subset U$, and $W$ is an open such that $K\subset W\subset \overline{W}\subset V$, then y the previous constructions, the following diagram is commutative:
    % https://q.uiver.app/#q=WzAsOCxbMCwwLCJcXG1hdGhjYWx7R30oXFxvdmVybGluZXtVfSkiXSxbMCwyLCJcXG1hdGhjYWx7R30oXFxvdmVybGluZXtWfSkiXSxbMSwxLCJcXHZhcmluamxpbVxcbGltaXRzX3tLXFxzdWJzZXQgVX1cXG1hdGhjYWx7R30oXFxvdmVybGluZXtVfSkiXSxbMiwwLCJcXHZhcnByb2psaW1cXGxpbWl0c197VVxcc3Vwc2V0IEsnfVxcbWF0aGNhbHtHfShLJykiXSxbMiwyLCJcXHZhcnByb2psaW1cXGxpbWl0c197Vlxcc3Vwc2V0IEsnfVxcbWF0aGNhbHtHfShLJykiXSxbMiwxLCJcXHZhcmluamxpbVxcbGltaXRzX3tLXFxzdWJzZXQgVX1cXHZhcnByb2psaW1cXGxpbWl0c197VVxcc3Vwc2V0IEsnfVxcbWF0aGNhbHtHfShLJylcXG1hdGhjYWx7R30oSycpIl0sWzMsMSwiXFxtYXRoY2Fse0d9KFxcb3ZlcmxpbmV7V30pIl0sWzQsMSwiXFx2YXJpbmpsaW1cXGxpbWl0c197S1xcc3Vic2V0IFV9XFxtYXRoY2Fse0d9KFxcb3ZlcmxpbmV7VX0pIl0sWzAsMV0sWzAsMl0sWzEsMl0sWzAsM10sWzEsNF0sWzMsNV0sWzIsNV0sWzQsNV0sWzMsNl0sWzQsNl0sWzUsNl0sWzYsN11d
\[\begin{tikzcd}
	{\mathcal{G}(\overline{U})} && {\varprojlim\limits_{U\supset K'}\mathcal{G}(K')} \\
	& {\varinjlim\limits_{K\subset U}\mathcal{G}(\overline{U})} & {\varinjlim\limits_{K\subset U}\varprojlim\limits_{U\supset K'}\mathcal{G}(K')} & {\mathcal{G}(\overline{W})} & {\varinjlim\limits_{K\subset U}\mathcal{G}(\overline{U})} \\
	{\mathcal{G}(\overline{V})} && {\varprojlim\limits_{V\supset K'}\mathcal{G}(K')}
	\arrow[from=1-1, to=1-3]
	\arrow[from=1-1, to=2-2]
	\arrow[from=1-1, to=3-1]
	\arrow[from=1-3, to=2-3]
	\arrow[from=1-3, to=2-4]
	\arrow[from=2-2, to=2-3]
	\arrow[from=2-3, to=2-4]
	\arrow[from=2-4, to=2-5]
	\arrow[from=3-1, to=2-2]
	\arrow[from=3-1, to=3-3]
	\arrow[from=3-3, to=2-3]
	\arrow[from=3-3, to=2-4]
\end{tikzcd}\]

In particular to conclude that the following square is commutative:

% https://q.uiver.app/#q=WzAsNCxbMCwwLCJcXG1hdGhjYWx7R30oXFxvdmVybGluZXtVfSkiXSxbMCwyLCJcXG1hdGhjYWx7R30oXFxvdmVybGluZXtWfSkiXSxbMiwwLCJcXHZhcmluamxpbVxcbGltaXRzX3tLXFxzdWJzZXQgVX1cXG1hdGhjYWx7R30oXFxvdmVybGluZXtVfSkiXSxbMiwyLCJcXHZhcmluamxpbVxcbGltaXRzX3tLXFxzdWJzZXQgVX1cXG1hdGhjYWx7R30oXFxvdmVybGluZXtVfSkiXSxbMCwxXSxbMCwyXSxbMSwzXSxbMiwzXV0=
\[\begin{tikzcd}
	{\mathcal{G}(\overline{U})} && {\varinjlim\limits_{K\subset U}\mathcal{G}(\overline{U})} \\
	\\
	{\mathcal{G}(\overline{V})} && {\varinjlim\limits_{K\subset U}\mathcal{G}(\overline{U})}
	\arrow[from=1-1, to=1-3]
	\arrow[from=1-1, to=3-1]
	\arrow[from=1-3, to=3-3]
	\arrow[from=3-1, to=3-3]
\end{tikzcd}\]
    One has to check that the morphism $\mathcal{G}(\overline{V})\to \varprojlim\limits_{V\supset K'}\mathcal{G}(K')\to \mathcal{G}(\overline{W})\to \varinjlim\limits_{K\subset U}\mathcal{G}(\overline{U})$ is the canonical inclusion. By the compatibility condition of the universal property of the projective limit $\mathcal{G}(\overline{V})\to \varprojlim\limits_{V\supset K'}\mathcal{G}(K')\to \mathcal{G}(\overline{W})$ is the map $\mathcal{G}(\overline{V}\supset \overline{W})$. And the maps $\mathcal{G}(\overline{V})\to \mathcal{G}(\overline{W})\to \varinjlim\limits_{K\subset U}\mathcal{G}(\overline{U})$ is the canonical inclusion by the compatibility condition of the universal property of the injective limit.

    However by the universal property of the inductive limit the only morphism that make this diagram commute is the identity.\\

    For the other direction, one keeps the same notations. By the previous constructions the following diagram is commutative:
    % https://q.uiver.app/#q=WzAsNyxbMCwwLCJcXHZhcnByb2psaW1cXGxpbWl0c197VVxcc3Vwc2V0IEsnfVxcbWF0aGNhbHtHfShLJykiXSxbMCwzLCJcXHZhcnByb2psaW1cXGxpbWl0c197Vlxcc3Vwc2V0IEsnfVxcbWF0aGNhbHtHfShLJykiXSxbMCwxLCJcXHZhcmluamxpbVxcbGltaXRzX3tLXFxzdWJzZXQgVX1cXHZhcnByb2psaW1cXGxpbWl0c197VVxcc3Vwc2V0IEsnfVxcbWF0aGNhbHtHfShLJykiXSxbMywxLCJcXG1hdGhjYWx7R30oXFxvdmVybGluZXtXfSkiXSxbNCwxLCJcXHZhcmluamxpbVxcbGltaXRzX3tLXFxzdWJzZXQgVX1cXG1hdGhjYWx7R30oXFxvdmVybGluZXtVfSkiXSxbNCwyLCJcXHZhcmluamxpbVxcbGltaXRzX3tLXFxzdWJzZXQgVX1cXHZhcnByb2psaW1cXGxpbWl0c197VVxcc3Vwc2V0IEsnfVxcbWF0aGNhbHtHfShLJykiXSxbMywyLCJcXHZhcnByb2psaW1cXGxpbWl0c197V1xcc3Vwc2V0IEsnfVxcbWF0aGNhbHtHfShLJykiXSxbMCwyXSxbMSwyXSxbMCwzXSxbMSwzXSxbMiwzXSxbMyw0XSxbNCw1XSxbMyw2XSxbNiw1XV0=
$$\begin{tikzcd}
	{\varprojlim\limits_{U\supset K'}\mathcal{G}(K')} \\
	{\varinjlim\limits_{K\subset U}\varprojlim\limits_{U\supset K'}\mathcal{G}(K')} &&& {\mathcal{G}(\overline{W})} & {\varinjlim\limits_{K\subset U}\mathcal{G}(\overline{U})} \\
	&&& {\varprojlim\limits_{W\supset K'}\mathcal{G}(K')} & {\varinjlim\limits_{K\subset U}\varprojlim\limits_{U\supset K'}\mathcal{G}(K')} \\
	{\varprojlim\limits_{V\supset K'}\mathcal{G}(K')}
	\arrow[from=1-1, to=2-1]
	\arrow[from=1-1, to=2-4]
	\arrow[from=2-1, to=2-4]
	\arrow[from=2-4, to=2-5]
	\arrow[from=2-4, to=3-4]
	\arrow[from=2-5, to=3-5]
	\arrow[from=3-4, to=3-5]
	\arrow[from=4-1, to=2-1]
	\arrow[from=4-1, to=2-4]
\end{tikzcd}$$

    One can remark that the map $\varprojlim\limits_{V\supset K'}\mathcal{G}(K')\to \mathcal{G}(\overline{W})\to \varprojlim\limits_{W\supset K'}\mathcal{G}(K')$ is the canonical map. By the universal property of the projective limit, it's enough to show that it's true when postcomposed by the maps (for K' a compact of $W$)$\varprojlim\limits_{W\supset K'}\mathcal{G}(K')\to\mathcal{G}(K')$. By construction $\mathcal{G}(\overline{W})\to \varprojlim\limits_{W\supset K'}\mathcal{G}(K')\to\mathcal{G}(K')=\mathcal{G}(\overline{W}\subset K')$, so the statment is a compatibility condition in the universal property of the projective limit.

    Then by the compatibility condition of the universal property of the inductive limit the map $\varprojlim\limits_{W\supset K'}\mathcal{G}(K')\to \varprojlim\limits_{W\supset K'}\mathcal{G}(K')\to \varinjlim\limits_{K\subset U}\varprojlim\limits_{U\supset K'}\mathcal{G}(K')$ is also the canonical map. Then the following diagram is commutative: 
    % https://q.uiver.app/#q=WzAsNCxbMCwwLCJcXHZhcnByb2psaW1cXGxpbWl0c197VVxcc3Vwc2V0IEsnfVxcbWF0aGNhbHtHfShLJykiXSxbMCwyLCJcXHZhcnByb2psaW1cXGxpbWl0c197Vlxcc3Vwc2V0IEsnfVxcbWF0aGNhbHtHfShLJykiXSxbMiwwLCJcXHZhcmluamxpbVxcbGltaXRzX3tLXFxzdWJzZXQgVX1cXHZhcnByb2psaW1cXGxpbWl0c197VVxcc3Vwc2V0IEsnfVxcbWF0aGNhbHtHfShLJykiXSxbMiwyLCJcXHZhcmluamxpbVxcbGltaXRzX3tLXFxzdWJzZXQgVX1cXHZhcnByb2psaW1cXGxpbWl0c197VVxcc3Vwc2V0IEsnfVxcbWF0aGNhbHtHfShLJykiXSxbMCwyXSxbMiwzXSxbMSwzXSxbMCwxXV0=
    $$\begin{tikzcd}
	{\varprojlim\limits_{U\supset K'}\mathcal{G}(K')} && {\varinjlim\limits_{K\subset U}\varprojlim\limits_{U\supset K'}\mathcal{G}(K')} \\
	\\
	{\varprojlim\limits_{V\supset K'}\mathcal{G}(K')} && {\varinjlim\limits_{K\subset U}\varprojlim\limits_{U\supset K'}\mathcal{G}(K')}
	\arrow[from=1-1, to=1-3]
	\arrow[from=1-1, to=3-1]
	\arrow[from=1-3, to=3-3]
	\arrow[from=3-1, to=3-3]
    \end{tikzcd}$$

    An again by uniqueness it must be the identity, that concludes the proof.

        \item Let $U$ be an open of $X$. One has to check that $\mathcal{F}(U)\to \varprojlim\limits_{U\supset K \text{ compact}}\varinjlim\limits_{K\subset U' \text{ open}}\mathcal{F}(U')$ is an isomorphism.

        One can apply the previous item with the compact $K=\{p\}$ (for $p\in X$) and then get that $\alpha_*$ preserve the stlaks. The fact that $\alpha^*$ preserves the stalks is straightforward by definition. Then the map $\mathcal{F}(U)\to \varprojlim\limits_{U\supset K \text{ compact}}\varinjlim\limits_{K\subset U' \text{ open}}\mathcal{F}(U')$ is an isomorphism once restricted to stalks, and because the two are sheaves on $X$ they are then isomorphics.
    \end{itemize}
\end{proof}