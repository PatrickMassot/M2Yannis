
Let $X$ be a locally compact Hausdorf space.

\section{Sheaves}


\begin{definition}\label{def:pre_sheaves}
    A presheave on $X$ is a contravariant functor from the category of open sets of $X$ to abélian groups.
\end{definition}

\begin{definition}\label{def:stalk}
    \uses{def:pre_sheaves}
    If $\mathcal{F}$ is a presheaf on $X$ and $p\in X$ then the stalk of $\mathcal{F}$ at $p$ is the abelian group $\mathcal{F}_p:=\varinjlim\limits_{p\in U\text{ open}}\mathcal{F}(U)$.
\end{definition}

\begin{definition}\label{def:sheaves}
    \uses{def:pre_sheaves}
    If $\mathcal{F}$ is a presheaf on $X$, it is said to be a sheaf if for any $U\subset X$ open and any covering family of $U$ $(U_a)_{a\in A}$ one has the exact sequence:
    \begin{equation}\label{axiom:Sh}
        0\to \mathcal{F}(U)\to \prod\limits_{a\in A}\mathcal{F}(U_a)\to \prod\limits_{a,b\in A}\mathcal{F}(U_a\cap U_b)
    \end{equation}
\end{definition}

\section{$\mathcal{K}$-sheaves}

\begin{definition}\label{def:K_pre_sheaves}
    A $\mathcal{K}$-presheave on $X$ is a contravariant functor from the category of compact sets of $X$ to abélian groups.
\end{definition}

\begin{definition}\label{def:K_stalk}
    \uses{def:K_pre_sheaves}
    If $\mathcal{F}$ is a $\mathcal{K}$-presheaf on $X$ and $p\in X$ then the stalk of $\mathcal{F}$ at $p$ is the abelian group $\mathcal{F}_p:=\varinjlim\limits_{p\in K\text{ compact}}\mathcal{F}(K)=\mathcal{F}(\{p\})$.  
\end{definition}

\begin{definition}\label{def:K_sheaves}
    \uses{def:K_pre_sheaves}
    If $\mathcal{F}$ is a $\mathcal{K}$-presheaf on $X$, it is said to be a $\mathcal{K}$-sheaf if the folowing conditions are satisfied:\begin{itemize}
    \item\begin{equation}\label{axiom:Ksh1}
        \mathcal{F}(\varnothing)=0
    \end{equation}
    \item For $K_1$ and $K_2$ two comapcts of $X$ the folowing sequence is exact:\begin{equation}\label{axiom:Ksh2}
         0\to\mathcal{F}(K_1\cup K_2)\to \mathcal{F}(K_1)\bigoplus\mathcal{F}(K_2)\to \mathcal{F}(K_1\cap K_2) 
    \end{equation}
    \item For any compact $K$ of $X$, the following natural morphism is an isomorphism\begin{equation}\label{axiom:Ksh3}
        \varinjlim\limits_{K\subset U\text{ open relatively compact}}\mathcal{F}(\overline{U})\to \mathcal{F}(K)
    \end{equation}
\end{itemize}
\end{definition}

\begin{remark}
    \eqref{axiom:Ksh3} is well defined because if $K$ is a compact subset of $X$, then for $x\in K$ let $U_x$ be an open neighborhood relatively compact (wich exists by local compactness), the family $(U_x)\ind{x}{K}$ covers $K$ then one can extract a finite cover of it : $U_1,\ldots U_n$ and then $\cup_{i=1}^n U_i$ is an open neighborhood, and a finite union of relatively comapct, then it's relatively compact. ($\overline{\cup_{i=1}^n U_i}=\cup_{i=1}^n \overline{U_i}$)
\end{remark}


\section{Technical lemmas}

\begin{lemma}\label{lem:cofinal_syst_of_inter_compact}
    If $K_1,\ldots K_n$ are comapcts of $X$ then $\{U_1\cap\ldots\cap U_n\}_{U_i\supset K_i\text{ open in }X}$ is a cofinal system of neighborhoods of $K_1\cap \ldots \cap K_n$.
\end{lemma}

\begin{proof}
    It's the theorem \text{IsCompact.nhdsSet\_inter\_eq}  in the File Mathlib/Topology/Separation.lean and the use of Filter.HasBasis.inf in the file  Mathlib.Order.Filter.Bases

   %Let $U_i$ be a relatively comapct open neighborhood of $K_i$ and $U=\cup_{i=1}^nU_i$. Then $\overline{U}$ is compact \\

   %If $n=2$, let $V$ be a neighborhood of $K_1\cap K_2$. By considering $U\cap V$, one ca assume that $V\subset U$.

   %If the result is true for $n-1$, 
\end{proof}

\begin{lemma}\label{lem:equiv_of_adj}
    If $\mathcal{C}$ and $\mathcal{D}$ are two categories, $F:\mathcal{C}\to \mathcal{D}$ and $G:\mathcal{D}\to \mathcal{C}$ two functors such that $(F,G)$ is an adjoint pair. Then for $(F,G)$ to be an equivalence of category, it's enough to have that thes canonical naturals transformations $\text{id}_{\mathcal{D}}\Rightarrow F\circ G$ and $G\circ F\Rightarrow \text{id}_{\mathcal{D}}$ are isomorphisms.
\end{lemma}

\begin{proof}
    CategoryTheory.Adjunction.toEquivalence in mathlib
\end{proof}

%\begin{lemma}\label{lem:a_nommer}
%    \uses{def:K_pre_sheaves nh}
%    If $(K_a)_{a\in A}$ is a filtered directed system of comapcts substes of $X$, and $\mathcal{F}$ a $\mathcal{K}$-presheaf satisfying\eqref{axiom:Ksh3}, then \[\varinjlim\limits_{a\in A}\mathcal{F}(K_a)\to \mathcal{F}(\bigcap\limits_{a\in A}K_a)\] is an isomorphism.
%\end{lemma}
%\begin{proof}
%    TODO
%\end{proof}

\section{Equivalence of category}

\begin{definition}\label{def:adj_kprshv_and_prshv}
    \uses{def:K_pre_sheaves,def:pre_sheaves}
    \begin{itemize}
        \item If $\mathcal{F}$ is a presheaf then let $\alpha^*\mathcal{F}$ be the $\mathcal{K}$-presheaf:\[K\mapsto \varinjlim\limits_{K\subset U \text{ open}}\mathcal{F}(U)\]
        \item If $\mathcal{G}$ is a $\mathcal{K}$-presheaf then let $\alpha_*\mathcal{G}$ be the presheaf:\[ U\mapsto \varprojlim\limits_{U\supset K \text{ compact}}\mathcal{G}(K)\]
    \end{itemize}
\end{definition}

\begin{proposition}\label{pro:adj_kprshv_and_prshv}
    \uses{def:adj_kprshv_and_prshv}
    The pair $(\alpha^*,\alpha_*)$ is an adjonit pair.
\end{proposition}

\begin{proof}
    \begin{itemize}
        \item Let $\tau$ be an element of $\hom(\alpha^*\mathcal{F},\mathcal{G})$. It's the data of morphism $\tau_K$ for $K$ a compact of $X$ such that for any $K$ and $K'$ compacts 
        % https://q.uiver.app/#q=WzAsNCxbMCwwLCJcXHZhcmluamxpbVxcbGltaXRzX3tLXFxzdWJzZXQgVX1cXG1hdGhjYWx7Rn0oVSkiXSxbMCwyLCJcXHZhcmluamxpbVxcbGltaXRzX3tLJ1xcc3Vic2V0IFV9XFxtYXRoY2Fse0Z9KFUpIl0sWzIsMCwiXFxtYXRoY2Fse0d9KEspIl0sWzIsMiwiXFxtYXRoY2Fse0d9KEsnKSJdLFswLDFdLFswLDIsIlxcdGF1X0siXSxbMSwzLCJcXHRhdV97Syd9Il0sWzIsM11d
\begin{equation}\label{nat_trans_afg}\begin{tikzcd}
	{\varinjlim\limits_{K\subset U}\mathcal{F}(U)} && {\mathcal{G}(K)} \\
	\\
	{\varinjlim\limits_{K'\subset U}\mathcal{F}(U)} && {\mathcal{G}(K')}
	\arrow["{\tau_K}", from=1-1, to=1-3]
	\arrow[from=1-1, to=3-1]
	\arrow[from=1-3, to=3-3]
	\arrow["{\tau_{K'}}", from=3-1, to=3-3]
\end{tikzcd}\end{equation} is a commutative square. Then for any $U$ and $V$ opens,  by composing with the commutative square % https://q.uiver.app/#q=WzAsNCxbMiwwLCJcXHZhcmluamxpbVxcbGltaXRzX3tLXFxzdWJzZXQgVX1cXG1hdGhjYWx7Rn0oVSkiXSxbMiwyLCJcXHZhcmluamxpbVxcbGltaXRzX3tLJ1xcc3Vic2V0IFV9XFxtYXRoY2Fse0Z9KFUpIl0sWzAsMCwiXFxtYXRoY2Fse0Z9KFUpIl0sWzAsMiwiXFxtYXRoY2Fse0Z9KFYpIl0sWzAsMV0sWzIsMF0sWzIsM10sWzMsMV1d
$$\begin{tikzcd}
	{\mathcal{F}(U)} && {\varinjlim\limits_{K\subset U}\mathcal{F}(U)} \\
	\\
	{\mathcal{F}(V)} && {\varinjlim\limits_{K'\subset U}\mathcal{F}(U)}
	\arrow[from=1-1, to=1-3]
	\arrow[from=1-1, to=3-1]
	\arrow[from=1-3, to=3-3]
	\arrow[from=3-1, to=3-3]
\end{tikzcd}$$ one get's a commutative square :% https://q.uiver.app/#q=WzAsNCxbMCwwLCJcXG1hdGhjYWx7Rn0oVSkiXSxbMCwyLCJcXG1hdGhjYWx7Rn0oVikiXSxbMiwwLCJcXG1hdGhjYWx7R30oSykiXSxbMiwyLCJcXG1hdGhjYWx7R30oSycpIl0sWzAsMV0sWzIsM10sWzEsM10sWzAsMl1d
\begin{equation}\label{data_adj}\begin{tikzcd}
	{\mathcal{F}(U)} && {\mathcal{G}(K)} \\
	\\
	{\mathcal{F}(V)} && {\mathcal{G}(K')}
	\arrow[from=1-1, to=1-3]
	\arrow[from=1-1, to=3-1]
	\arrow[from=1-3, to=3-3]
	\arrow[from=3-1, to=3-3]
\end{tikzcd}\end{equation}. Conversely such data give rise (by taking the limit over $U$ and $V$) to a commutative square such as in \eqref{nat_trans_afg}

    \item On the other hand if one takes the limit over $K$ and $K'$ one get's a commutative square 
    % https://q.uiver.app/#q=WzAsNCxbMCwwLCJcXG1hdGhjYWx7Rn0oVSkiXSxbMCwyLCJcXG1hdGhjYWx7Rn0oVikiXSxbMiwwLCJcXHZhcnByb2psaW1cXGxpbWl0c197S1xcc3Vic2V0IFV9XFxtYXRoY2Fse0d9KEspIl0sWzIsMiwiXFx2YXJwcm9qbGltXFxsaW1pdHNfe0tcXHN1YnNldCBWfVxcbWF0aGNhbHtHfShLKSJdLFswLDFdLFsyLDNdLFsxLDNdLFswLDJdXQ==
$$\begin{tikzcd}
	{\mathcal{F}(U)} && {\varprojlim\limits_{K\subset U}\mathcal{G}(K)} \\
	\\
	{\mathcal{F}(V)} && {\varprojlim\limits_{K\subset V}\mathcal{G}(K)}
	\arrow[from=1-1, to=1-3]
	\arrow[from=1-1, to=3-1]
	\arrow[from=1-3, to=3-3]
	\arrow[from=3-1, to=3-3]
\end{tikzcd}$$ (that allow to recover the previous one in the same as before) wich is the data of an element of $\hom(\mathcal{F},\alpha_*\mathcal{G})$.
\end{itemize}
    Then the elements of $\hom(\alpha^*\mathcal{F},\mathcal{G})$ and $\hom(\mathcal{F},\alpha_*\mathcal{G})$ are both obtained by a natural construction (in $\mathcal{F}$ and $\mathcal{G}$) aplied to \eqref{data_adj}.
\end{proof}

\begin{lemma}\label{lem:adj_kshv_and_shv}
    \uses{def:adj_kprshv_and_prshv}
    \begin{itemize}
        \item $\alpha^*$ send sheaves to $\mathcal{K}$-sheaves
        \item $\alpha^*$ send $\mathcal{K}$-sheaves to sheaves
        \item The reistrictions obtained still form an adjoint pair between shaeves and $\mathcal{K}$-sheaves.
    \end{itemize}
\end{lemma}

\begin{proof}
    \uses{def:K_sheaves, def:sheaves, lem:cofinal_syst_of_inter_compact}
    \begin{itemize}
        \item Let $\mathcal{F}$ be a sheaf. The condition $\varnothing \subset U$ is always satisfied and $\varnothing$ is minimal among open subset for the inclusion then $(\alpha^*)(\mathcal{F})(\varnothing)=\mathcal{F}(\varnothing)$. One can apply the sheaf condition to the empty family and obtain the exact sequence $0\to \mathcal{F}(\varnothing)\to \Pi_{\varnothing}=0$, and then \eqref{axiom:Ksh1}.\\

        Let $K_1,K_2$ be two of compacts of $X$, let $U_1,U_2$ be a two opens such that $K_i\subset U_i$ for all $i$. Then the sheaf condition gives an exact sequence (because for abelian groups the product is the direct sum) $0\to \mathcal{F}(U_1\cup U_2)\to \mathcal{F}(U_1)\bigoplus\mathcal{F}(U_2)\to \mathcal{F}(U_1\cap U_2)$. The injective limits are exacts then taking the limits over those opens gives an exact sequence: 
        \begin{equation}\label{lim_of_sheaf_cond}
            0\to \varinjlim\limits_{K_i\subset U_i}\mathcal{F}(U_1\cup U_2)\to \varinjlim\limits_{K_i\subset U_i}\mathcal{F}(U_1)\bigoplus\mathcal{F}(U_2)\to \varinjlim\limits_{K_i\subset U_i}\mathcal{F}(U_1\cap U_2)
        \end{equation}

        An open $U$ contains $K_1\cup K_2$ if and only if it's of the form $U_1\cup U_2$ with $K_i\subset U_i$ (one can take $U_1=U_2=U$ for the direct implication), then by definition $\varinjlim\limits_{K_i\subset U_i}\mathcal{F}(U_1\cup U_2)=\alpha^*\mathcal{F}(K_1\cup K_2)$.\\
        
        The injective limit commute with the direct product, then: $$\varinjlim\limits_{K_i\subset U_i}\mathcal{F}(U_1)\bigoplus\mathcal{F}(U_2)=(\varinjlim\limits_{K_i\subset U_i}\mathcal{F}(U_1))\bigoplus(\varinjlim\limits_{K_i\subset U_i}\mathcal{F}(U_2))=\alpha^*\mathcal{F}(K_1)\bigoplus\alpha^*\mathcal{F}(K_2)$$.\\
        
        By the lemma \ref{lem:cofinal_syst_of_inter_compact} the limit $\varinjlim\limits_{K_i\subset U_i}\mathcal{F}(U_1\cap U_2)$ compute the same thing as $\varinjlim\limits_{K_1\cap K_2\subset U}\mathcal{F}(U)=\alpha^*\mathcal{F}(K_1\cap K_2)$.\\

        Then the exact sequence \eqref{lim_of_shaef_cond} is in fact \eqref{axiom:Ksh2}.\\ \\


        Let $K$ be a compact, $U$ a relatively comapct open such that $K\subset U$ and $V$ an open suche that $\bar{U}\subset V$ then $K\subset V$. Conversely if $V$ is an open containing $K$, then $K$ is a comapct of $V$ (locally compact as $X$ is) and then admits an open neighborhood $U$ relatively compact (in $V$).Thus (because the two limits are over the same set) one has the equality \[\varinjlim\limits_{K\subset U \text{open relatively compact}}\varinjlim\limits_{\bar{U}\subset V\text{ open}}\mathcal{F}(V)=\varinjlim\limits_{K\subset U\ \text{open}}\mathcal{F}(V)\]. Wich rewrite by definition as $\varinjlim\limits_{K\subset U\ \text{open relatively compact}}\alpha^*\mathcal{F}(\bar{U})=\alpha^*\mathcal{F}(V)$ i.e. \eqref{axiom:Ksh3}




        \item
        \item A morphisme between two ($\mathcal{K}$-)sheaves is by definition is by definition a morphisme between the two underling ($\mathcal{K}$-)presheaves then, the natural equality $\hom_{\text{Sh}}(\alpha^*\mathcal{F},\mathcal{G})=\hom_{\text{Sh}}(\mathcal{F},\alpha_*\mathcal{G})$ is a consequence of \ref{pro:adj_kprshv_and_prshv}
    \end{itemize}
\end{proof}

\begin{lemma}\label{lem:kshv_equiv_shv}
    \uses{def:K_sheaves, def:sheaves, lem:adj_kshv_and_shv}
    The previous adjoint pair give rise to an equivalence of category between shaeves and $\mathcal{K}$-sheaves
\end{lemma}

\begin{proof}
    \uses{def:K_sheaves, def:sheaves, lem:adj_kshv_and_shv, lem:equiv_of_adj}
    By using \ref{lem:equiv_of_adj}, it's enough to show that for any sheaf $\mathcal{F}$ and $\Kc$-sheaf $\mathcal{G}$, the natural maps $\mathcal{F}\to \alpha_*\alpha^*\mathcal{U}$ and $\alpha^*\alpha_*\mathcal{G}\to \mathcal{G}$ are isomorphism. The fact of being a natural isomorphism can be checked locally.\begin{itemize}
        \item Let $U$ be an open of $X$. One has to check that $\mathcal{F}(U)\to \varprojlim\limits_{U\supset K \text{ compact}}\varinjlim\limits_{K\subset U' \text{ open}}\mathcal{F}(U')$ is an isomorphism.
        \item
    \end{itemize}
\end{proof}