Let $X$ and $Y$ be two loccaly compacts hausdorf spaces and $f:X\to Y$ be a continuous map.

\section{For Sheaves}

\begin{definition}\label{def:pushforward_of_sheaf}
    \uses{def:sheaves}
    If $\mathcal{F}$ is a pre-sheaf over $X$, then the rule $U\mapsto \mathcal{F}(f^{-1}(U))$ defines a pre-shaef over $Y$.

    The functor obtained is denoted $f_*$ and named the pushforward by $f$.

    $f_*$ send sheaves over $X$ into sheaves over $Y$.
\end{definition}

\begin{proof}
    Let $(U_a)\ind{a}{A}$ be a family of opens of $Y$. Then one ca apply the sheaf condition of $\mathcal{F}$ with the family of opens of $X$: $(f^{-1}(U_a))\ind{a}{A}$. The result is the exact sequence: $$0\to \mathcal{F}(\bigcup\limits\ind{a}{A}f^{-1}(U_a))\to \prod\limits_{a\in A}\mathcal{F}(f^{-1}(U_a))\to \prod\limits_{a,b\in A}\mathcal{F}(f^{-1}(U_a)\cap f^{-1}(U_b))$$. 
    On the other hand, the inverse image commute with union and intersections, then the previous exact sequence rewrites to $$0\to f_*\mathcal{F}(\bigcup\limits\ind{a}{A}U_a)\to \prod\limits_{a\in A}f_*\mathcal{F}(U_a)\to \prod\limits_{a,b\in A}f_*\mathcal{F}(U_a\cap U_b)$$. 
    In other words, $f_*\mathcal{F}$ is a sheaf.
\end{proof}

\begin{definition}\label{def:pullback_of_sheaf}
    \uses{def:sheaves}
    If $\mathcal{F}$ is a pre-sheaf over $Y$, then the rule $U\mapsto \varinjlim\limits_{f(U)\subset V}\mathcal{F}(V)$ defines a pre-sheaf over $Y$.

    If $\mathcal{F}$ is a sheaf, the sheafification of the previous pre-sheaf is a denoted $f^*\mathcal{F}$ and called the pullback by $f$.

\end{definition}

\begin{definition}\label{def:exceptional_pushforward_of_sheaf}
    If $f:X\to Y$ is the inclusion of an open subset, the exceptional pushforward by $f$: $f_!$ is defined by $f_!\mathcal{F}(U)$ being the subset of $f_*\mathcal{F}(U)$ of sections that vanish over a neighborhood of $Y-X$.

    It send the sheaves over $X$ into the sheaves over $Y$
\end{definition}

\begin{proof}
    \uses{def:pushforward_of_sheaf}
    Let $U\supset V$ be two opens of $Y$ and $h$ be an element of $f_!\mathcal{F}(U)$, then $h$ is an element of $\mathcal{F}(U\cap X)$ such that there is a $W$ open that contains $Y\backslash X$ and such that $h|_{U\cap W\cap X}=0$. Then  $0=h|_V|_{U\cap W\cap X}=h|_{V\cap U\cap W\cap X}=h|_{V\cap W\cap X}$ so $h|_V$ is in $f_!\mathcal{F}(V)$. So $f_!\mathcal{F}$ is well defined.

    Let $(U_a)\ind{a}{A}$ be a family of opens of $Y$. The map $f_!\mathcal{F}(\bigcup\limits\ind{a}{A}U_a\cap X)\to\Pi\limits\ind{a}{A}f_!\mathcal{F}(U_a)$ is a reistriction of an injective map (because of the sheaf condition of $f_*\mathcal{F}$), then it's also injective.

    Let $(h_a)$ be an element of the kernel of $\prod\limits_{a\in A}f_!\mathcal{F}(U_a)\to \prod\limits_{a,b\in A}f_!\mathcal{F}(U_a\cap U_b)$. By the sheaf condition of $f_*\mathcal{F}$, it's of the form $(h|_{U_a})$ with $h\in f_*\mathcal{F}(\bigciup\limits\ind{a}{A}U_a)$. To conclude the sheaf condition for $f_!\mathcal{F}$ one has to check that $h$ is $f_!\mathcal{F}(\bigciup\limits\ind{a}{A}U_a)$.

    By definition for any $a\in A$ there is an open $V_a$ of $Y$ that contains $Y\backslash X$ and such that $h_a|_{U_a\cap V_a\cap X}=0$. So for all $a\in A\, h_{U_a\cap V_a\cap X}=0$. Let $V$ be the union of the $V_a$, it contains $Y\backslash X$. the reistriction of $h|_V$ to all $V_a\cap X$ are $0$, then by the first part of the sheaf condition, $h|_{V\cap X}$ is also $0$, then $h$ is in $f_!\mathcal{F}(\bigciup\limits\ind{a}{A}U_a)$.    
    
\end{proof}

\section{For $\mathcal{K}$-sheaves}

Let's assume that $f$ is proper.

\begin{definition}\label{def:pushforward_of_K-sheaf}
    \uses{def:K_sheaves}
    If $\mathcal{F}$ is a pre-$\mathcal{K}$-sheaf over $X$, then the rule $K\mapsto \mathcal{F}(f^{-1}(K))$ defines a pre-$\mathcal{K}$-sheaf over $Y$.

    The functor obtained is denoted $f_*$ and named the pushforward by $f$.

    $f_*$ send $\mathcal{K}$-sheaves over $X$ into $\mathcal{K}$-sheaves over $Y$.
\end{definition}

\begin{proof}
    TODO
\end{proof}

%\begin{definition}\label{def:pullback_of_Homotopy-K-sheaf}
%    If $f$ is injective, and $\mathcal{F}^{\bullet}$ a Homotopy-$\mathcal{K}$-sheaf over $Y$, then the rule $K\mapsto \mathcal{F}^{\bullet}(f(K))$ defines a Homotopy-$\mathcal{K}$-sheaf over $X$
%\end{definition}

%\begin{proof}
%    \uses{lem:lim_comm_inter_for_hksh}

%\end{proof}

% j'en ai vraiment besoin?
