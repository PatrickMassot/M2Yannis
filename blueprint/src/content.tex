% In this file you should put the actual content of the blueprint.
% It will be used both by the web and the print version.
% It should *not* include the \begin{document}
%
% If you want to split the blueprint content into several files then
% the current file can be a simple sequence of \input. Otherwise It
% can start with a \section or \chapter for instance.


\chapter{Presheaves and sheaves}

Let $X$ be a locally compact Hausdorf space.

\section{Sheaves}


\begin{definition}
    \label{def:pre_sheaves}
    A presheave on $X$ is a contravariant functor from the category of open sets of $X$ to abélian groups.
\end{definition}

\begin{definition}
    \label{def:stalk}
    \uses{def:pre_sheaves}
    If $\mathcal{F}$ is a presheaf on $X$ and $p\in X$ then the stalk of $\mathcal{F}$ at $p$ is the abelian group $\mathcal{F}_p:=\varinjlim\limits_{p\in U\text{ open}}\mathcal{F}(U)$.
\end{definition}

\begin{definition}
    \label{def:sheaves}
    \uses{def:pre_sheaves}
    If $\mathcal{F}$ is a presheaf on $X$, it is said to be a sheaf if for any $U\subset X$ open and any covering family of $U$ $(U_a)_{a\in A}$ one has the exact sequence:
    \begin{equation}\label{axiom:Sh}
        0\to \mathcal{F}(U)\to \prod\limits_{a\in A}\mathcal{F}(U_a)\to \prod\limits_{a,b\in A}F(U_a\cap U_b)
    \end{equation}
\end{definition}

\section{$\mathcal{K}$-sheaves}

\begin{definition}
    \label{def:K_pre_sheaves}
    A $\mathcal{K}$-presheave on $X$ is a contravariant functor from the category of compact sets of $X$ to abélian groups.
\end{definition}

\begin{definition}
    \label{def:K_stalk}
    \uses{def:K_pre_sheaves}
    If $\mathcal{F}$ is a $\mathcal{K}$-presheaf on $X$ and $p\in X$ then the stalk of $\mathcal{F}$ at $p$ is the abelian group $\mathcal{F}_p:=\varinjlim\limits_{p\in K\text{ compact}}\mathcal{F}(K)=\mathcal{F}(\{p\})$.  
\end{definition}

\begin{definition}
    \label{def:K_sheaves}
    \uses{def:K_pre_sheaves}
    If $\mathcal{F}$ is a $\mathcal{K}$-presheaf on $X$, it is said to be a $\mathcal{K}$-sheaf if the folowing conditions are satisfied:\begin{itemize}
    \item\begin{equation}\label{axiom:Ksh1}
        \mathcal{F}(\varnothing)=0
    \end{equation}
    \item For $K_1$ and $K_2$ two comapcts of $X$ the folowing sequence is exact:\begin{equation}\label{axiom:Ksh2}
         0\to\mathcal{F}(K_1\cup K_2)\to \mathcal{F}(K_1)\bigoplus\mathcal{F}(K_2)\to \mathcal{F}(K_1\cap K_2) 
    \end{equation}
    \item Pour tout compact $K$ de $X$, le morphisme naturel suivant est un isomorphisme \begin{equation}\label{axiom:Ksh3}
        \varinjlim\limits_{K\subset U\text{ open relatively compact}}\mathcal{F}(\overline{U})\to \mathcal{F}(K)
        %Dans un espace localement compact tout compact admet un voisinage ouvert relativemrnnt compact
    \end{equation}
\end{itemize}
\end{definition}

\begin{remark}
    \eqref{axiom:Ksh3} is well defined because if $K$ is a compact subset of $X$, then for $x\in K$ let $U_x$ be an open neighborhood relatively compact (wich exists by local compactness), the family $(u_x)\ind{x}{K}$ covers $K$ then one can extract a finite cover of it : $U_1,\ldots U_n$ and then $\cup_{i=1}^n U_i$ is an open neighborhood, and a finite union of relatively comapct, then it's relatively compact. ($\overline{\cup_{i=1}^n U_i}=\cup_{i=1}^n \overline{U_i}$)
\end{remark}


\section{Technical lemmas}

\begin{lemma}
    \label{lem:cofinal_syst_of_inter_compact}
    If $K_1,\ldots K_n$ are comapcts of $X$ then $\{U_1\cap\ldots\cap U_n\}_{U_i\supset K_i\text{ open in }X}$ is a cofinal system of neighborhoods of $K_1\cap \ldots K_n$.
\end{lemma}

\begin{proof}
   Let $U_i$ be a relatively comapct open neighborhood of $K_i$ and $U=\cup_{i=1}^nU_i$. Then $\overline{U}$ is compact \\

   If $n=2$, let $V$ be a neighborhood of $K_1\cap K_2$

   If the result is true for $n-1$, 
\end{proof}

\begin{lemma}
    \label{lem:equiv_of_adj}
    \lean{CategoryTheory.Adjunction.toEquivalence}
    \leanok
    If $\mathcal{C}$ and $\mathcal{D}$ are two categories, $F:\mathcal{C}\to \mathcal{D}$ and $G:\mathcal{D}\to \mathcal{C}$ two functors such that $(F,G)$ is an adjoint pair. Then for $(F,G)$ to be an equivalence of category, it's enough to have that thes canonical naturals transformations $\text{id}_{\mathcal{D}}\Rightarrow F\circ G$ and $G\circ F\Rightarrow \text{id}_{\mathcal{D}}$ are isomorphisms.
\end{lemma}

\begin{proof}
    \leanok
    TODO%CategoryTheory.Adjunction.toEquivalence dans mathlib
\end{proof}

\begin{lemma}
    \label{lem:a_nommer}
    \uses{def:K_pre_sheaves nh}
    If $(K_a)_{a\in A}$ is a filtered directed system of comapcts substes of $X$, and $\mathcal{F}$ a $\mathcal{K}$-presheaf satisfying \eqref{axiom:Ksh3}, then \[\varinjlim\limits_{a\in A}\mathcal{F}(K_a)\to \mathcal{F}(\bigcap\limits_{a\in A}K_a)\] is an isomorphism.
\end{lemma}
\begin{proof}
    TODO
\end{proof}

\section{Equivalence of category}

\begin{definition}
    \label{def:adj_kprshv_and_prshv}
    \uses{def:K_pre_sheaves ,def:pre_sheaves}
    \begin{itemize}
        \item If $\mathcal{F}$ is a presheaf then let $\alpha^*\mathcal{F}$ ne the $\mathcal{K}$-presheaf :\[K\mapsto \varinjlim\limits_{K\subset U \text{ open}}\mathcal{F}(U)\]
        \item If $\mathcal{G}$ is a $\mathcal{K}$-presheaf then let $\alpha_*\mathcal{G}$ ne the presheaf :\[ U\mapsto \varprojlim\limits_{U\supset K \text{ compact}}\mathcal{F}(K)\]
    \end{itemize}
\end{definition}

\begin{proposition}
    \label{pro:adj_kprshv_and_prshv}
    \uses{def:adj_kprshv_and_prshv}
    The pair $(\alpha^*,\alpha_*)$ is an adjonit pair.
\end{proposition}

\begin{proof}
    TODO
\end{proof}

\begin{lemma}
    \label{lem:adj_kshv_and_shv}
    \uses{def:adj_kprshv_and_prshv}
    \begin{itemize}
        \item $\alpha^*$ send sheaves to $\mathcal{K}$-sheaves
        \item $\alpha^*$ send $\mathcal{K}$-sheaves to sheaves
        \item The reistrictions obtained still form an adjoint pair.
    \end{itemize}
    The previous adjoint pair give rise to an adjoint pair between shaeves and $\mathcal{K}$-sheaves
\end{lemma}

\begin{proof}
    \uses{def:K_sheaves, def:sheaves, lem:cofinal_syst_of_inter_compact}
    TODO
\end{proof}

\begin{lemma}
    \label{lem:kshv_equiv_shv}
    \uses{def:K_sheaves, def:sheaves, lem:adj_kshv_and_shv}
    The previous adjoint pair give rise to an equivalence of category between shaeves and $\mathcal{K}$-sheaves
\end{lemma}

\begin{proof}
    \uses{def:K_sheaves, def:sheaves, lem:adj_kshv_and_shv, lem:equiv_of_adj}

\end{proof}








\chapter{Homotopy sheaves}

\chapter{Pushforward, exceptional pushforward, and pullback}

\chapter{\v{C}ech cohomology}

\chapter{Purehomotopy $\mathcal{K}$-sheaves}

\chapter{Poincaré–Lefschetz duality}

\chapter{Homotopy colimits}

\chapter{Homotopy colimits of pure homotopy $\mathcal{K}$-sheaves}

\chapter{Steenrod homology}