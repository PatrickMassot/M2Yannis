% In this file you should put the actual content of the blueprint.
% It will be used both by the web and the print version.
% It should *not* include the \begin{document}
%
% If you want to split the blueprint content into several files then
% the current file can be a simple sequence of \input. Otherwise It
% can start with a \section or \chapter for instance.

\begin{definition}\label{def:total_complex}
    If $A_0^{\bullet}\to\ldots \to A_n^{\bullet}$ is a sequence of maps (with $f_i:A_i\to A_{i+1}$ )of complex such that the composition of two consecutive maps is $0$, then let's denote $[A_0^{\bullet}\to\ldots \to A_n^{\bullet-n}]$ the total complex of this double complex, defined by the following data:\begin{itemize}
        \item The object in degré $k$ is $\bigoplus\limits_{i=0}^nA_i[-i]^n$
        \item The diferential is given by the matrix $\begin{pmatrix}
            d_{A_0} & 0 & \ldots & \ldots & 0\\
            f_0 & -d_{A_1} & \ldots & \ldots & 0\\
            0 & f_1 & \ldots & \ldots & 0\\
            \vdots & \vdots & \ddots &  & \vdots\\
            \vdots & \vdots & & (-1)^{n-1}d_{A_{n-1}} & 0\\
            0 & 0 & \ldots & f_{n-1} & (-1)^nd_{A_n}\\
            \end{pmatrix}$
    \end{itemize}
\end{definition}

\begin{proof}
    One needs to check that the matrix square is $0$. Let $M$ be this matrix and $(i,j)$ be integers.

    \[M^2[i,j]=\sum\limits_{k=1}^nM[i,k]M[k,j]= M[i,i] M[i,j]+M[i,i-1]M[i-1,j]\]

    One can distinguish four cases:\begin{itemize}

        \item If $j$ is not in $\{i-2,i-1,i,\}$ then the two terms are $0$. 
        \item If $j=i$, then $M^2[i,j]=((-1)^id_{A_i})^2+0=0$. 
        \item If $j=i-1$ then $M^2[i,j]0+(-1)^id_{A_i}\circ f_i+ (-1)^{i+1}d_{A_{i+1}}\circ f_i=0$ because $f_i$ is a morphism of complex.
        \item If $j=i-2$ then $M^2[i,j]=f_{i}\circ f_{i-1}=0$.
    \end{itemize}
\end{proof}

\begin{remark}
    In particular, if $f:A^{\bullet}\to B^{\bullet}$ is a morphism of complex then $[A^{\bullet}\to B^{\bullet-1}]$ is the cone of the morphism f.
\end{remark}

\begin{lemma}\label{lem:q-iso_iff_cone_acyclic}
    A morphism of complex $f:A^{\bullet}\to B^{\bullet}$ is a quasi isomorphism if and only if, its cone is acyclic.
\end{lemma}

\begin{proof}
    One get's a short exact sequence $0\to B^{\bullet}[-1]\to [A^{\bullet}\to B^{\bullet-1}]\to A^{\bullet} \to 0$
    by using the canonical inclusion and projection over the direct sum. The long exact sequence induced in cohomology is then: \[\ldots H^{k-1}A^{\bullet} \to H^k B^{\bullet}[-1]\to H^k[A^{\bullet}\to B^{\bullet-1}]\to H^k A^{\bullet}\to H^{k+1}B^{\bullet}[-1]\ldots\]
    By using the fact that $H^k B^{\bullet}[-1]=H^{k-1}B^{\bullet}$ one gets:  \[\ldots\to  H^{k-1} A^{\bullet} \to H^{k-1} B^{\bullet}\to H^k[A^{\bullet}\to B^{\bullet-1}]\to H^k A^{\bullet}\to H^kB^{\bullet}\ldots\]

    And then the statement is straightforward by reading the exact sequence.
\end{proof}

\begin{lemma}\label{lem:complex_total_of_three_is_acyclic}
    If a complex $[A^{\bullet}\to B^{\bullet-1}\to C^{\bullet-2}]$ is acyclic then there is a long exact sequence \[\ldots \to H^k A\to H^k B\to H^k C\to \ldots\]
\end{lemma}

\begin{proof}
    \uses{lem:q-iso_iff_cone_acyclic}
    One can see that by construction there is a canonical isomorphism of complexess: $[A^{\bullet}\to B^{\bullet-1}\to C^{\bullet-2}]=[A^{\bullet}\to [B^{\bullet}\to C^{\bullet-1}]^{\bullet-1}]$.

    Then by the previous lemma: $A^{\bullet}\to [B^{\bullet}\to C^{\bullet-1}]$ is a quasi isomorphism. One can then rewrite the long exact sequence in cohomology givent by the short exact sequence $0\to C^{\bullet}[-1]\to [B^{\bullet}\to C^{\bullet-1}]\to B^{\bullet} \to 0$ wich is (as in the previous lemma): \[\ldots\to  H^{k-1} B^{\bullet} \to H^{k-1} C^{\bullet}\to H^k[B^{\bullet}\to C^{\bullet-1}]\to H^k B^{\bullet}\to H^kC^{\bullet}\to \ldots\]. 

    The result is then a long exact sequence : \[\ldots\to  H^{k-1} B^{\bullet} \to H^{k-1} C^{\bullet}\to H^kA^{\bullet}\to H^k B^{\bullet}\to H^k C^{\bullet}\to\ldots\]

\end{proof}

\chapter{Presheaves and sheaves}


Let $X$ be a locally compact Hausdorf space.

\section{Sheaves}


\begin{definition}\label{def:pre_sheaves}
    A presheave on $X$ is a contravariant functor from the category of open sets of $X$ to abélian groups.
\end{definition}

\begin{definition}\label{def:stalk}
    \uses{def:pre_sheaves}
    If $\mathcal{F}$ is a presheaf on $X$ and $p\in X$ then the stalk of $\mathcal{F}$ at $p$ is the abelian group $\mathcal{F}_p:=\varinjlim\limits_{p\in U\text{ open}}\mathcal{F}(U)$.
\end{definition}

\begin{definition}\label{def:sheaves}
    \uses{def:pre_sheaves}
    If $\mathcal{F}$ is a presheaf on $X$, it is said to be a sheaf if for any $U\subset X$ open and any covering family of $U$ $(U_a)_{a\in A}$ one has the exact sequence:
    \begin{equation}\label{axiom:Sh}
        0\to \mathcal{F}(U)\to \prod\limits_{a\in A}\mathcal{F}(U_a)\to \prod\limits_{a,b\in A}\mathcal{F}(U_a\cap U_b)
    \end{equation}
\end{definition}

\section{$\mathcal{K}$-sheaves}

\begin{definition}\label{def:K_pre_sheaves}
    A $\mathcal{K}$-presheave on $X$ is a contravariant functor from the category of compact sets of $X$ to abélian groups.
\end{definition}

\begin{definition}\label{def:K_stalk}
    \uses{def:K_pre_sheaves}
    If $\mathcal{F}$ is a $\mathcal{K}$-presheaf on $X$ and $p\in X$ then the stalk of $\mathcal{F}$ at $p$ is the abelian group $\mathcal{F}_p:=\varinjlim\limits_{p\in K\text{ compact}}\mathcal{F}(K)=\mathcal{F}(\{p\})$.  
\end{definition}

\begin{definition}\label{def:K_sheaves}
    \uses{def:K_pre_sheaves}
    If $\mathcal{F}$ is a $\mathcal{K}$-presheaf on $X$, it is said to be a $\mathcal{K}$-sheaf if the folowing conditions are satisfied:\begin{itemize}
    \item\begin{equation}\label{axiom:Ksh1}
        \mathcal{F}(\varnothing)=0
    \end{equation}
    \item For $K_1$ and $K_2$ two comapcts of $X$ the folowing sequence is exact:\begin{equation}\label{axiom:Ksh2}
         0\to\mathcal{F}(K_1\cup K_2)\to \mathcal{F}(K_1)\bigoplus\mathcal{F}(K_2)\to \mathcal{F}(K_1\cap K_2) 
    \end{equation}
    \item For any compact $K$ of $X$, the following natural morphism is an isomorphism\begin{equation}\label{axiom:Ksh3}
        \varinjlim\limits_{K\subset U\text{ open relatively compact}}\mathcal{F}(\overline{U})\to \mathcal{F}(K)
    \end{equation}
\end{itemize}
\end{definition}

\begin{remark}
    \eqref{axiom:Ksh3} is well defined because if $K$ is a compact subset of $X$, then for $x\in K$ let $U_x$ be an open neighborhood relatively compact (wich exists by local compactness), the family $(U_x)\ind{x}{K}$ covers $K$ then one can extract a finite cover of it : $U_1,\ldots U_n$ and then $\cup_{i=1}^n U_i$ is an open neighborhood, and a finite union of relatively comapct, then it's relatively compact. ($\overline{\cup_{i=1}^n U_i}=\cup_{i=1}^n \overline{U_i}$)
\end{remark}


\section{Technical lemmas}

\begin{lemma}\label{lem:cofinal_syst_of_inter_compact}
    If $K_1,\ldots K_n$ are comapcts of $X$ then $\{U_1\cap\ldots\cap U_n\}_{U_i\supset K_i\text{ open in }X}$ is a cofinal system of neighborhoods of $K_1\cap \ldots \cap K_n$.
\end{lemma}

\begin{proof}
    It's the theorem \text{IsCompact.nhdsSet\_inter\_eq}  in the File Mathlib/Topology/Separation.lean and the use of Filter.HasBasis.inf in the file  Mathlib.Order.Filter.Bases

   %Let $U_i$ be a relatively comapct open neighborhood of $K_i$ and $U=\cup_{i=1}^nU_i$. Then $\overline{U}$ is compact \\

   %If $n=2$, let $V$ be a neighborhood of $K_1\cap K_2$. By considering $U\cap V$, one ca assume that $V\subset U$.

   %If the result is true for $n-1$, 
\end{proof}

\begin{lemma}\label{lem:equiv_of_adj}
    If $\mathcal{C}$ and $\mathcal{D}$ are two categories, $F:\mathcal{C}\to \mathcal{D}$ and $G:\mathcal{D}\to \mathcal{C}$ two functors such that $(F,G)$ is an adjoint pair. Then for $(F,G)$ to be an equivalence of category, it's enough to have that thes canonical naturals transformations $\text{id}_{\mathcal{D}}\Rightarrow F\circ G$ and $G\circ F\Rightarrow \text{id}_{\mathcal{D}}$ are isomorphisms.
\end{lemma}

\begin{proof}
    CategoryTheory.Adjunction.toEquivalence in mathlib
\end{proof}

%\begin{lemma}\label{lem:a_nommer}
%    \uses{def:K_pre_sheaves nh}
%    If $(K_a)_{a\in A}$ is a filtered directed system of comapcts substes of $X$, and $\mathcal{F}$ a $\mathcal{K}$-presheaf satisfying\eqref{axiom:Ksh3}, then \[\varinjlim\limits_{a\in A}\mathcal{F}(K_a)\to \mathcal{F}(\bigcap\limits_{a\in A}K_a)\] is an isomorphism.
%\end{lemma}
%\begin{proof}
%    TODO
%\end{proof}

\section{Equivalence of category}

\begin{definition}\label{def:adj_kprshv_and_prshv}
    \uses{def:K_pre_sheaves,def:pre_sheaves}
    \begin{itemize}
        \item If $\mathcal{F}$ is a presheaf then let $\alpha^*\mathcal{F}$ be the $\mathcal{K}$-presheaf:\[K\mapsto \varinjlim\limits_{K\subset U \text{ open}}\mathcal{F}(U)\]
        \item If $\mathcal{G}$ is a $\mathcal{K}$-presheaf then let $\alpha_*\mathcal{G}$ be the presheaf:\[ U\mapsto \varprojlim\limits_{U\supset K \text{ compact}}\mathcal{G}(K)\]
    \end{itemize}
\end{definition}

\begin{proposition}\label{pro:adj_kprshv_and_prshv}
    \uses{def:adj_kprshv_and_prshv}
    The pair $(\alpha^*,\alpha_*)$ is an adjonit pair.
\end{proposition}

\begin{proof}
    \begin{itemize}
        \item Let $\tau$ be an element of $\hom(\alpha^*\mathcal{F},\mathcal{G})$. It's the data of morphism $\tau_K$ for $K$ a compact of $X$ such that for any $K$ and $K'$ compacts 
        % https://q.uiver.app/#q=WzAsNCxbMCwwLCJcXHZhcmluamxpbVxcbGltaXRzX3tLXFxzdWJzZXQgVX1cXG1hdGhjYWx7Rn0oVSkiXSxbMCwyLCJcXHZhcmluamxpbVxcbGltaXRzX3tLJ1xcc3Vic2V0IFV9XFxtYXRoY2Fse0Z9KFUpIl0sWzIsMCwiXFxtYXRoY2Fse0d9KEspIl0sWzIsMiwiXFxtYXRoY2Fse0d9KEsnKSJdLFswLDFdLFswLDIsIlxcdGF1X0siXSxbMSwzLCJcXHRhdV97Syd9Il0sWzIsM11d
\begin{equation}\label{nat_trans_afg}\begin{tikzcd}
	{\varinjlim\limits_{K\subset U}\mathcal{F}(U)} && {\mathcal{G}(K)} \\
	\\
	{\varinjlim\limits_{K'\subset U}\mathcal{F}(U)} && {\mathcal{G}(K')}
	\arrow["{\tau_K}", from=1-1, to=1-3]
	\arrow[from=1-1, to=3-1]
	\arrow[from=1-3, to=3-3]
	\arrow["{\tau_{K'}}", from=3-1, to=3-3]
\end{tikzcd}\end{equation} is a commutative square. Then for any $U$ and $V$ opens,  by composing with the commutative square % https://q.uiver.app/#q=WzAsNCxbMiwwLCJcXHZhcmluamxpbVxcbGltaXRzX3tLXFxzdWJzZXQgVX1cXG1hdGhjYWx7Rn0oVSkiXSxbMiwyLCJcXHZhcmluamxpbVxcbGltaXRzX3tLJ1xcc3Vic2V0IFV9XFxtYXRoY2Fse0Z9KFUpIl0sWzAsMCwiXFxtYXRoY2Fse0Z9KFUpIl0sWzAsMiwiXFxtYXRoY2Fse0Z9KFYpIl0sWzAsMV0sWzIsMF0sWzIsM10sWzMsMV1d
$$\begin{tikzcd}
	{\mathcal{F}(U)} && {\varinjlim\limits_{K\subset U}\mathcal{F}(U)} \\
	\\
	{\mathcal{F}(V)} && {\varinjlim\limits_{K'\subset U}\mathcal{F}(U)}
	\arrow[from=1-1, to=1-3]
	\arrow[from=1-1, to=3-1]
	\arrow[from=1-3, to=3-3]
	\arrow[from=3-1, to=3-3]
\end{tikzcd}$$ one get's a commutative square :% https://q.uiver.app/#q=WzAsNCxbMCwwLCJcXG1hdGhjYWx7Rn0oVSkiXSxbMCwyLCJcXG1hdGhjYWx7Rn0oVikiXSxbMiwwLCJcXG1hdGhjYWx7R30oSykiXSxbMiwyLCJcXG1hdGhjYWx7R30oSycpIl0sWzAsMV0sWzIsM10sWzEsM10sWzAsMl1d
\begin{equation}\label{data_adj}\begin{tikzcd}
	{\mathcal{F}(U)} && {\mathcal{G}(K)} \\
	\\
	{\mathcal{F}(V)} && {\mathcal{G}(K')}
	\arrow[from=1-1, to=1-3]
	\arrow[from=1-1, to=3-1]
	\arrow[from=1-3, to=3-3]
	\arrow[from=3-1, to=3-3]
\end{tikzcd}\end{equation}. Conversely such data give rise (by taking the limit over $U$ and $V$) to a commutative square such as in \eqref{nat_trans_afg}

    \item On the other hand if one takes the limit over $K$ and $K'$ one get's a commutative square 
    % https://q.uiver.app/#q=WzAsNCxbMCwwLCJcXG1hdGhjYWx7Rn0oVSkiXSxbMCwyLCJcXG1hdGhjYWx7Rn0oVikiXSxbMiwwLCJcXHZhcnByb2psaW1cXGxpbWl0c197S1xcc3Vic2V0IFV9XFxtYXRoY2Fse0d9KEspIl0sWzIsMiwiXFx2YXJwcm9qbGltXFxsaW1pdHNfe0tcXHN1YnNldCBWfVxcbWF0aGNhbHtHfShLKSJdLFswLDFdLFsyLDNdLFsxLDNdLFswLDJdXQ==
$$\begin{tikzcd}
	{\mathcal{F}(U)} && {\varprojlim\limits_{K\subset U}\mathcal{G}(K)} \\
	\\
	{\mathcal{F}(V)} && {\varprojlim\limits_{K\subset V}\mathcal{G}(K)}
	\arrow[from=1-1, to=1-3]
	\arrow[from=1-1, to=3-1]
	\arrow[from=1-3, to=3-3]
	\arrow[from=3-1, to=3-3]
\end{tikzcd}$$ (that allow to recover the previous one in the same as before) wich is the data of an element of $\hom(\mathcal{F},\alpha_*\mathcal{G})$.
\end{itemize}
    Then the elements of $\hom(\alpha^*\mathcal{F},\mathcal{G})$ and $\hom(\mathcal{F},\alpha_*\mathcal{G})$ are both obtained by a natural construction (in $\mathcal{F}$ and $\mathcal{G}$) aplied to \eqref{data_adj}.
\end{proof}

\begin{lemma}\label{lem:adj_kshv_and_shv}
    \uses{def:adj_kprshv_and_prshv}
    \begin{itemize}
        \item $\alpha^*$ send sheaves to $\mathcal{K}$-sheaves
        \item $\alpha^*$ send $\mathcal{K}$-sheaves to sheaves
        \item The reistrictions obtained still form an adjoint pair between shaeves and $\mathcal{K}$-sheaves.
    \end{itemize}
\end{lemma}

\begin{proof}
    \uses{def:K_sheaves, def:sheaves, lem:cofinal_syst_of_inter_compact}
    \begin{itemize}
        \item Let $\mathcal{F}$ be a sheaf. The condition $\varnothing \subset U$ is always satisfied and $\varnothing$ is minimal among open subset for the inclusion then $(\alpha^*)(\mathcal{F})(\varnothing)=\mathcal{F}(\varnothing)$. One can apply the sheaf condition to the empty family and obtain the exact sequence $0\to \mathcal{F}(\varnothing)\to \Pi_{\varnothing}=0$, and then \eqref{axiom:Ksh1}.\\

        Let $K_1,K_2$ be two of compacts of $X$, let $U_1,U_2$ be a two opens such that $K_i\subset U_i$ for all $i$. Then the sheaf condition gives an exact sequence (because for abelian groups the product is the direct sum) $0\to \mathcal{F}(U_1\cup U_2)\to \mathcal{F}(U_1)\bigoplus\mathcal{F}(U_2)\to \mathcal{F}(U_1\cap U_2)$. The injective limits are exacts then taking the limits over those opens gives an exact sequence: 
        \begin{equation}\label{lim_of_sheaf_cond}
            0\to \varinjlim\limits_{K_i\subset U_i}\mathcal{F}(U_1\cup U_2)\to \varinjlim\limits_{K_i\subset U_i}\mathcal{F}(U_1)\bigoplus\mathcal{F}(U_2)\to \varinjlim\limits_{K_i\subset U_i}\mathcal{F}(U_1\cap U_2)
        \end{equation}

        An open $U$ contains $K_1\cup K_2$ if and only if it's of the form $U_1\cup U_2$ with $K_i\subset U_i$ (one can take $U_1=U_2=U$ for the direct implication), then by definition $\varinjlim\limits_{K_i\subset U_i}\mathcal{F}(U_1\cup U_2)=\alpha^*\mathcal{F}(K_1\cup K_2)$.\\
        
        The injective limit commute with the direct product, then: $$\varinjlim\limits_{K_i\subset U_i}\mathcal{F}(U_1)\bigoplus\mathcal{F}(U_2)=(\varinjlim\limits_{K_i\subset U_i}\mathcal{F}(U_1))\bigoplus(\varinjlim\limits_{K_i\subset U_i}\mathcal{F}(U_2))=\alpha^*\mathcal{F}(K_1)\bigoplus\alpha^*\mathcal{F}(K_2)$$.\\
        
        By the lemma \ref{lem:cofinal_syst_of_inter_compact} the limit $\varinjlim\limits_{K_i\subset U_i}\mathcal{F}(U_1\cap U_2)$ compute the same thing as $\varinjlim\limits_{K_1\cap K_2\subset U}\mathcal{F}(U)=\alpha^*\mathcal{F}(K_1\cap K_2)$.\\

        Then the exact sequence \eqref{lim_of_shaef_cond} is in fact \eqref{axiom:Ksh2}.\\ \\


        Let $K$ be a compact, $U$ a relatively comapct open such that $K\subset U$ and $V$ an open suche that $\bar{U}\subset V$ then $K\subset V$. Conversely if $V$ is an open containing $K$, then $K$ is a comapct of $V$ (locally compact as $X$ is) and then admits an open neighborhood $U$ relatively compact (in $V$).Thus (because the two limits are over the same set) one has the equality \[\varinjlim\limits_{K\subset U \text{open relatively compact}}\varinjlim\limits_{\bar{U}\subset V\text{ open}}\mathcal{F}(V)=\varinjlim\limits_{K\subset U\ \text{open}}\mathcal{F}(V)\]. Wich rewrite by definition as $\varinjlim\limits_{K\subset U\ \text{open relatively compact}}\alpha^*\mathcal{F}(\bar{U})=\alpha^*\mathcal{F}(V)$ i.e. \eqref{axiom:Ksh3}




        \item
        \item A morphisme between two ($\mathcal{K}$-)sheaves is by definition is by definition a morphisme between the two underling ($\mathcal{K}$-)presheaves then, the natural equality $\hom_{\text{Sh}}(\alpha^*\mathcal{F},\mathcal{G})=\hom_{\text{Sh}}(\mathcal{F},\alpha_*\mathcal{G})$ is a consequence of \ref{pro:adj_kprshv_and_prshv}
    \end{itemize}
\end{proof}

\begin{lemma}\label{lem:kshv_equiv_shv}
    \uses{def:K_sheaves, def:sheaves, lem:adj_kshv_and_shv}
    The previous adjoint pair give rise to an equivalence of category between shaeves and $\mathcal{K}$-sheaves
\end{lemma}

\begin{proof}
    \uses{def:K_sheaves, def:sheaves, lem:adj_kshv_and_shv, lem:equiv_of_adj}
    By using \ref{lem:equiv_of_adj}, it's enough to show that for any sheaf $\mathcal{F}$ and $\Kc$-sheaf $\mathcal{G}$, the natural maps $\mathcal{F}\to \alpha_*\alpha^*\mathcal{U}$ and $\alpha^*\alpha_*\mathcal{G}\to \mathcal{G}$ are isomorphism. The fact of being a natural isomorphism can be checked locally.\begin{itemize}
        \item Let $U$ be an open of $X$. One has to check that $\mathcal{F}(U)\to \varprojlim\limits_{U\supset K \text{ compact}}\varinjlim\limits_{K\subset U' \text{ open}}\mathcal{F}(U')$ is an isomorphism.
        \item
    \end{itemize}
\end{proof}

\chapter{Homotopy sheaves}

\begin{definition}\label{def:cohomology_of_k-prsh}
    \uses{def:K_pre_sheaves}
    Let $\mathcal{F}^{\bullet}$ be complex of $\mathcal{K}$-presheaves then taking the cohomology defines a $\mathcal{K}$-presheaf denoted $H^{\bullet}\mathcal{F}^{\bullet}$.
\end{definition}

\begin{definition}\label{def:quasi_iso_of_complex_of_k-prsh}
    \uses{def:cohomology_of_k-prsh}
    A morphisme of complex of $\mathcal{K}$-presheave $\mathcal{F}^{\bullet}\to\mathcal{G}^{\bullet}$ if it induces isomorphisms $H^i\mathcal{F}^{\bullet}\to H^i\mathcal{G}^{\bullet}$ for all $i$.
\end{definition}

\begin{definition}\label{def:homotopy_k_sheaf}
    \uses{def:K_pre_sheaves, def:total_complex}
    A complex of $\mathcal{K}$-presheaves $\mathcal{F}^{\bullet}$ is said to be a Homotopy-$\mathcal{K}$-sheave if the folowing conditions are satisfied:\begin{itemize}
        \item\begin{equation}\label{axiom:hKsh1}
            \mathcal{F}^{\bullet}(\varnothing) \text{ is an acyclic complex}
        \end{equation}
        \item For $K_1$ and $K_2$ two comapcts of $X$ the folowing complex is acyclic:\begin{equation}\label{axiom:hKsh2}
             [\mathcal{F}^{\bullet}(K_1\cup K_2)\to \mathcal{F}^{\bullet-1}(K_1)\bigoplus\mathcal{F}^{\bullet-1}(K_2)\to \mathcal{F}^{\bullet-2}(K_1\cap K_2) ]
        \end{equation}
        \item For any compact $K$ of $X$, the following natural morphism is a quasi-isomorphism \begin{equation}\label{axiom:hKsh3}
            \varinjlim\limits_{K\subset U\text{ open relatively compact}}\mathcal{F}^{\bullet}(\overline{U})\to \mathcal{F}^{\bullet}(K)
        \end{equation}
    \end{itemize}
\end{definition}

\begin{remark}\label{lem:Mayer_Vietoris}
    \uses{lem:complex_total_of_three_is_acyclic}
    By using \ref{lem:complex_total_of_three_is_acyclic}, \eqref{axiom:hKsh3} give rise to a "Mayer-Vietoris" long exact sequence: \[\ldots \to H^k \mathcal{F}^{\bullet}(K_1\cup K_2)\to H^k \mathcal{F}^{\bullet}(K_1)\bigoplus H^k\mathcal{F}^{\bullet}(K_2)\to H^k \mathcal{F}^{\bullet}(K_1\cap K_2)\to \ldots\]
\end{remark}

\begin{lemma}\label{lem:homotpy_k_sheaves_stable_by_extension}
    \uses{def:K_pre_sheaves,def:homotopy_k_sheaf}
    Let $\mathcal{F}^{\bullet}$ be a complex of $\mathcal{K}$-presheaves. If $\mathcal{F}^{\bullet}$ has a finite filtration whose associated graded is a Homotopy-$\mathcal{K}$-sheaf, then $\mathcal{F}^{\bullet}$ is a Homotopy-$\mathcal{K}$-sheaf.
\end{lemma}

\begin{proof}
    TODO
\end{proof}

\begin{lemma}\label{lem:first_non_zero_homology_of_homotopy_k_sheaf_is_k_sheaf}
    \uses{def:K_sheaves,def:homotopy_k_sheaf}
    If $\mathcal{F}^{\bullet}$ is a homotopy-$\mathcal{K}$-sheaf, and $H^{-1}\mathcal{F}^{\bullet}=0$ then $H^{0}\mathcal{F}^{\bullet}$ is a $\mathcal{K}$-sheaf
\end{lemma}

\begin{proof}
    \uses{lem:Mayer_Vietoris}
    \begin{itemize}
        \item $\mathcal{F}^{\bullet}(\varnothing)$ is acyclic then in particular it's cohomology in degre $0$ is $0$, then one gets \eqref{axiom:Ksh1}
        \item $H^{-1}\mathcal{F}^{\bullet}(K_1\cap K_2)=0$ then the first terms \ref{lem:Mayer_Vietoris} gives the exact sequence of \eqref{axiom:Ksh2}
        \item TODO
    \end{itemize}
\end{proof}





\chapter{Pushforward, exceptional pushforward, and pullback}

Let $X$ and $Y$ be two loccaly compacts hausdorf spaces and $f:X\to Y$ be a continuous map.

\section{For Sheaves}

\begin{definition}\label{def:pushforward_of_sheaf}
    \uses{def:sheaves}
    If $\mathcal{F}$ is a pre-sheaf over $X$, then the rule $U\mapsto \mathcal{F}(f^{-1}(U))$ defines a pre-shaef over $Y$.

    The functor obtained is denoted $f_*$ and named the pushforward by $f$.

    $f_*$ send sheaves over $X$ into sheaves over $Y$.
\end{definition}

\begin{proof}
    Let $(U_a)\ind{a}{A}$ be a family of opens of $Y$. Then one ca apply the sheaf condition of $\mathcal{F}$ with the family of opens of $X$: $(f^{-1}(U_a))\ind{a}{A}$. The result is the exact sequence: $$0\to \mathcal{F}(\bigcup\limits\ind{a}{A}f^{-1}(U_a))\to \prod\limits_{a\in A}\mathcal{F}(f^{-1}(U_a))\to \prod\limits_{a,b\in A}\mathcal{F}(f^{-1}(U_a)\cap f^{-1}(U_b))$$. 
    On the other hand, the inverse image commute with union and intersections, then the previous exact sequence rewrites to $$0\to f_*\mathcal{F}(\bigcup\limits\ind{a}{A}U_a)\to \prod\limits_{a\in A}f_*\mathcal{F}(U_a)\to \prod\limits_{a,b\in A}f_*\mathcal{F}(U_a\cap U_b)$$. 
    In other words, $f_*\mathcal{F}$ is a sheaf.
\end{proof}

\begin{definition}\label{def:pullback_of_sheaf}
    \uses{def:sheaves}
    If $\mathcal{F}$ is a pre-sheaf over $Y$, then the rule $U\mapsto \varinjlim\limits_{f(U)\subset V}\mathcal{F}(V)$ defines a pre-sheaf over $Y$.

    If $\mathcal{F}$ is a sheaf, the sheafification of the previous pre-sheaf is a denoted $f^*\mathcal{F}$ and called the pullback by $f$.

\end{definition}

\begin{definition}\label{def:exceptional_pushforward_of_sheaf}
    If $f:X\to Y$ is the inclusion of an open subset, the exceptional pushforward by $f$: $f_!$ is defined by $f_!\mathcal{F}(U)$ being the subset of $f_*\mathcal{F}(U)$ of sections that vanish over a neighborhood of $Y-X$.

    It send the sheaves over $X$ into the sheaves over $Y$
\end{definition}

\begin{proof}
    \uses{def:pushforward_of_sheaf}
    Let $U\supset V$ be two opens of $Y$ and $h$ be an element of $f_!\mathcal{F}(U)$, then $h$ is an element of $\mathcal{F}(U\cap X)$ such that there is a $W$ open that contains $Y\backslash X$ and such that $h|_{U\cap W\cap X}=0$. Then  $0=h|_V|_{U\cap W\cap X}=h|_{V\cap U\cap W\cap X}=h|_{V\cap W\cap X}$ so $h|_V$ is in $f_!\mathcal{F}(V)$. So $f_!\mathcal{F}$ is well defined.

    Let $(U_a)\ind{a}{A}$ be a family of opens of $Y$. The map $f_!\mathcal{F}(\bigcup\limits\ind{a}{A}U_a\cap X)\to\prod\limits\ind{a}{A}f_!\mathcal{F}(U_a)$ is a reistriction of an injective map (because of the sheaf condition of $f_*\mathcal{F}$), then it's also injective.

    Let $(h_a)$ be an element of the kernel of $\prod\limits_{a\in A}f_!\mathcal{F}(U_a)\to \prod\limits_{a,b\in A}f_!\mathcal{F}(U_a\cap U_b)$. By the sheaf condition of $f_*\mathcal{F}$, it's of the form $(h|_{U_a})$ with $h\in f_*\mathcal{F}(\bigcup\limits\ind{a}{A}U_a)$. To conclude the sheaf condition for $f_!\mathcal{F}$ one has to check that $h$ is $f_!\mathcal{F}(\bigcup\limits\ind{a}{A}U_a)$.

    By definition for any $a\in A$ there is an open $V_a$ of $Y$ that contains $Y\backslash X$ and such that $h_a|_{U_a\cap V_a\cap X}=0$. So for all $a\in A\, h_{U_a\cap V_a\cap X}=0$. Let $V$ be the union of the $V_a$, it contains $Y\backslash X$. the reistriction of $h|_V$ to all $V_a\cap X$ are $0$, then by the first part of the sheaf condition, $h|_{V\cap X}$ is also $0$, then $h$ is in $f_!\mathcal{F}(\bigcup\limits\ind{a}{A}U_a)$.    
    
\end{proof}

\section{For $\mathcal{K}$-sheaves}

Let's assume that $f$ is proper.

\begin{lemma}\label{lem:basis_of_open_of_fm1K}
    If $K$ is a compact of $Y$, then $\{f^{-1}(U)\}_{K\subset U}$ is a basis of open neighborhoods of $f^{-1}(K)$.
\end{lemma}

\begin{proof}
    TODO
\end{proof}

\begin{definition}\label{def:pushforward_of_K-sheaf}
    \uses{def:K_sheaves}
    If $\mathcal{F}$ is a pre-$\mathcal{K}$-sheaf over $X$, then the rule $K\mapsto \mathcal{F}(f^{-1}(K))$ defines a pre-$\mathcal{K}$-sheaf over $Y$.

    The functor obtained is denoted $f_*$ and named the pushforward by $f$.

    $f_*$ send $\mathcal{K}$-sheaves over $X$ into $\mathcal{K}$-sheaves over $Y$.
\end{definition}

\begin{proof}
    \uses{lem:basis_of_open_of_fm1K,def:pushforward_of_sheaf}
    By the lemma \ref{lem:basis_of_open_of_fm1K}, for $K$ a compact of $Y$, $\varinjlim\limits_{K\subset U \text{ open in Y}}\mathcal{F}(f^{-1}(U))$ computes $\varinjlim\limits_{f^-1(K)\subset U \text{ open in X}}\mathcal{F}(U)$. In other words $f_*(\alpha^*\mathcal{F})=\alpha^*f_*(\mathcal{F})$. 

    Then if $\mathcal{G}$ is a $\Kc$-sheaf, it's of the form $\alpha^*\mathcal{F}$ for $\mathcal{F}$ some sheaf. Then $f_*\mathcal{G}$ is isomorphic to $\alpha^*f_*(\mathcal{F})$ wich is a sheaf because of \ref{def:pushforward_of_sheaf} and \ref{lem:adj_kshv_and_shv}
\end{proof}

%\begin{definition}\label{def:pullback_of_Homotopy-K-sheaf}
%    If $f$ is injective, and $\mathcal{F}^{\bullet}$ a Homotopy-$\mathcal{K}$-sheaf over $Y$, then the rule $K\mapsto \mathcal{F}^{\bullet}(f(K))$ defines a Homotopy-$\mathcal{K}$-sheaf over $X$
%\end{definition}

%\begin{proof}
%    \uses{lem:lim_comm_inter_for_hksh}

%\end{proof}

% j'en ai vraiment besoin?


\chapter{\v{C}ech cohomology}

\section{\v{C}ech cohomology of sheaves}

\begin{definition}\label{def:chech_of_sh}
    If $X$ is a topological space, and $\mathcal{F}$ a sheaf over $X$, then let $\check{H}^{\bullet}(X;\mathcal{F}))$ be the \v{C}ech cohomology of $X$ with coeficient in $\mathcal{F}$
\end{definition}

\begin{definition}\label{def:chech_of_compact_sup_of_sh}
    If $X$ is a topological space, $K$ a comapct subset of $X$ and $\mathcal{F}$ a sheaf over $X$, then let $\check{H}_K^{\bullet}(X;\mathcal{F}))$ be the \v{C}ech cohomology of $X$ with support in $K$ with coeficient in $\mathcal{F}$
\end{definition}

\begin{definition}\label{def:pullback_for_chech}
    \uses{def:chech_of_compact_sup_of_sh, def:chech_of_sh}
    Let $f:X\to Y$ be a continuous map between topological spaces, and $\mathcal{F}$ a sheaf over $X$. $f$ induces a natural map $\check{H}^{\bullet}(Y;f_*\mathcal{F})\to \check{H}^{\bullet}(X;\mathcal{F})$.\\

    Moreover if $f$ is proper, one gets a natural map $\check{H}_c^{\bullet}(Y;f_*\mathcal{F})\to \check{H}_c^{\bullet}(X;\mathcal{F})$

\end{definition}

\begin{lemma}\label{lem:chech_comm_with_exceptional}
    \uses{def:exceptional_pushforward_of_sheaf}
    Let $f:X\to Y$ be an inclusion of open subset, then there is a natural isomorphism $f_!:\check{H}_c^{\bullet}(X;\mathcal{F})\to \check{H}^{\bullet}(Y;f_!\mathcal{F})$
\end{lemma}

\begin{proof}

\end{proof}

\section{\v{C}ech cohomology of complex of $\mathcal{K}$-sheaves}

\begin{definition}\label{def:chech_of_compelx_of_k_prsh}
    Let $\mathcal{F}^{\bullet}$ be a complex of $\mathcal{K}$-presheaves on a compact space $X$ then we define the \v{C}ech cohomology $\check{H}(X;\mathcal{F}^{\bullet})$ by TODO
\end{definition}

\begin{remark}\label{def:chech_of_k_prsh}
    By using the inclusion of $\mathcal{K}$-presheaves into complexes of $\mathcal{K}$-presheave, one get's a definition of \v{C}ech cohomology for $\mathcal{K}$-presheave.
\end{remark}

\begin{lemma}\label{lem:chech_of_acyclic_is_zero}
    Let $mathcal{F}^{\bullet}$ be an acyclic complex of $\mathcal{K}$-presheaves, then $\check{H}^k(X;mathcal{F}^{\bullet})=0$

\end{lemma}

\begin{proof}
    TODO
\end{proof}

\begin{lemma}\label{lem:chech_long_exact_sequence}
    \uses{def:chech_of_compelx_of_k_prsh}
    Let $0\to \mathcal{F}^{\bullet}\to \mathcal{G}^{\bullet}\to \mathcal{H}^{\bullet}\to 0$ be a short exact sequence of complex of $\mathcal{K}$-presheaves. Then there is a long exact seuquence in \v{c}ech cohomology: 

    \[\ldots\to \check{H}^k(X;\mathcal{F}^{\bullet})\to \check{H}^k(X;\mathcal{G}^{\bullet})\to \check{H}^k(X;\mathcal{H}^{\bullet})\to \ldots\]

\end{lemma}

\begin{proof}
    TODO
\end{proof}

\begin{lemma}\label{lem:chech_preserve_quasi_iso}
    \uses{def:quasi_iso_of_complex_of_k-prsh}
    If $\mathcal{F}^{\bullet}\to\mathcal{G}^{\bullet}$ is a quasi-isomorphism then the induced maps $\check{H}^i\mathcal{F}^{\bullet}\to \check{H}^i\mathcal{G}^{\bullet}$ are isomorphims.
\end{lemma}

\begin{proof}
    \uses{lem:q-iso_iff_cone_acyclic,lem:chech_long_exact_sequence,lem:chech_of_acyclic_is_zero}
    By \ref{lem:q-iso_iff_cone_acyclic}, the complex $[\mathcal{F}^{\bullet}\to \mathcal{G}^{\bullet-1}]$ is acyclic then by \ref{lem:chech_of_acyclic_is_zero}, it's \v{c}ech cohomology is zero.

    But there is a short exact sequence $0\to \mathcal{G}^{\bullet}[-1]\to [\mathcal{F}^{\bullet}\to \mathcal{G}^{\bullet-1}]\to \mathcal{F}^{\bullet} \to 0$, then the long exact sequence induced by \ref{lem:chech_long_exact_sequence} gives the claimed isomorphisms.
\end{proof}

\begin{proposition}\label{prop:homotopy_k_sheaf_compute_chech}
    \uses{def:homotopy_k_sheaf,def:chech_of_compelx_of_k_prsh}
    Let $\mathcal{F}^{\bullet}$ be a complex of $\mathcal{K}$-presheaves that verify \eqref{axiom:hKsh1} and \eqref{axiom:hKsh2} then the canonical map $H^{\bullet}\mathcal{F}^{\bullet}\to \check{H}^{\bullet}(X;\mathcal{F}^{\bullet})$ is an isomorphism.
\end{proposition}

\begin{proof}
    TODO
    %use chech-cohomology and hshk (1 et 2)

\end{proof}

\section{\v{C}ech cohomology is determined by stalks}

\begin{lemma}\label{lem:chech_is_determined_by_stalks_1}
    \uses{def:cohomology_of_k-prsh, def:chech_of_compelx_of_k_prsh, def:K_stalk}
    Let $\mathcal{F}^{\bullet}$ be a complex of $\mathcal{K}$-presheaves that verify \eqref{axiom:Ksh3} and such that all the stalks are $0$ then $\check{H}^{\bullet}(X;\mathcal{F})=0$
    
\end{lemma}

\begin{proof}
    \uses{def:K_pre_sheaves}% axiom 3
\end{proof}

\begin{lemma}\label{lem:chech_is_determined_by_stalks_2}
    \uses{def:chech_of_compelx_of_k_prsh,def:homotopy_k_sheaf}
    Let $\mathcal{F}^{\bullet}$ be a complex of $\mathcal{K}$-presheaves that verify \eqref{axiom:hKsh3} and $H^i\mathcal{F}^{\bullet}=0$ for $i<<0$.
    
    Then if the stalks of $\mathcal{F}^{\bullet}$ are acyclics, $\check{H}^{\bullet}(X;\mathcal{F}^{\bullet})=0$

\end{lemma}

\begin{proof}
    \uses{lem:chech_is_determined_by_stalks_1,lem:chech_preserve_quasi_iso}
    TODO
\end{proof}


\begin{proposition}\label{prop:chech_is_determined_by_stalks_3}
    \uses{def:chech_of_compelx_of_k_prsh,def:homotopy_k_sheaf}
    Let $\mathcal{F}^{\bullet}$ and $\mathcal{G}^{\bullet}$ be complexes of $\mathcal{K}$-presheaves that verify \eqref{axiom:hKsh3} and $H^i\mathcal{F}^{\bullet}=H^i\mathcal{G}^{\bullet}=0$ for $i$ small enough.
    
    Then if a morphism $\mathcal{F}^{\bullet}\to \mathcal{G}^{\bullet}$ induces a quasi-isomorphism on stalks, $\check{H}^{\bullet}(X;\mathcal{F}^{\bullet})=\check{H}^{\bullet}(X;\mathcal{G}^{\bullet})$
\end{proposition}

\begin{proof}
    \uses{lem:chech_is_determined_by_stalks_2}
    %uses: 
    % équation A.4.9
\end{proof}

\chapter{Purehomotopy $\mathcal{K}$-sheaves}



\begin{definition}\label{def:pure_homotopy_k_sheaf}
    A homotopy $\mathcal{K}$-sheaf $\mathcal{F}^{\bullet}$ is said to be pure on $X$ if: \begin{itemize}
        \item For $p\in X$ and $i\neq 0$, $(H^i\mathcal{F}^{\bullet})_p=0$
        \item $H^i\mathcal{F}^{\bullet}=0$ for $i<<0$blocally on $X$: ie for all $p\in X$ there is an open neighbourhoud $U$ of $p$ and an integer $N$ such that for $i\leq N$ and $K\subset U$: $H^i\mathcal{F}^{\bullet}(K)=0$
    \end{itemize}
\end{definition}

\begin{lemma}\label{lem:csq_of_pure_homotopy_k_sheaf}
    Let $\mathcal{F}^{\bullet}$ be a pure-homotopy $\mathcal{K}$-sheaf. Then: \begin{itemize}
        \item For $i<0$ $H^i\mathcal{F}^{\bullet}=0$
        \item $H^0\mathcal{F}^{\bullet}$ is a $\mathcal{K}$-sheaf.
    \end{itemize}
\end{lemma}

\begin{proof}
    \uses{lem:chech_preserve_quasi_iso}
    %hSh_k_2
    TODO
\end{proof}

\begin{proposition}\label{prop:cech_of_pure_homotopy_k_sheaf}
    \uses{def:pure_homotopy_k_sheaf}
    Let $\mathcal{F}^{\bullet}$ be a pure-homotopy $\mathcal{K}$-sheaf. Then there is a canonical isomorphism:
    \[H^{\bullet}\mathcal{F}^{\bullet}(X)=\check{H}^{\bullet}(X;H^0\mathcal{F}^{\bullet})\].

    More generaly: Let $[\mathcal{F}_0^{\bullet}\to \ldots\mathcal{F}_n^{\bullet-n}]$  be a complex of pure-homotopy $\mathcal{K}$-sheaves, then there is a canonical isomorphism:

    \[H^{\bullet}[\mathcal{F}_0^{\bullet}(X)\to \ldots\mathcal{F}_n^{\bullet-n}(X)]=\check{H}^{\bullet}(X;[H^0\mathcal{F}_0^{\bullet}\to\ldots\to (H^0\mathcal{F}_n^{\bullet})[n]])\]
\end{proposition}

\begin{proof}
    \uses{lem:homotpy_k_sheaves_stable_by_extension,lem:csq_of_pure_homotopy_k_sheaf, prop:chech_is_determined_by_stalks_3,lem:first_non_zero_homology_of_homotopy_k_sheaf_is_k_sheaf,lem:chech_preserve_quasi_iso,prop:homotopy_k_sheaf_compute_chech}
    TODO
\end{proof}

\chapter{Poincaré–Lefschetz duality}

\input{Poincare–Lefschetz_duality}

\chapter{Homotopy colimits}

\section{Homotopy colimits}

\begin{definition}\label{def:homotopy_diagram}

\end{definition}

\begin{definition}\label{def:homotopy_colimit}
    \uses{def:homotopy_diagram}
\end{definition}

\section{Homotopy colimits of pure homotopy $\mathcal{K}$-sheaves}

\begin{lemma}\label{lem:hocolim_preserves_homotopy_k_sheaves}
    \uses{def:homotopy_colimit}
\end{lemma}

\begin{proof}
    \uses{lem:homotpy_k_sheaves_stable_by_extension}

\end{proof}

\begin{lemma}\label{lem:gluing_pure_homotopy_k_sheaves}
    \uses{lem:hocolim_preserves_homotopy_k_sheaves,def:pure_homotopy_k_sheaf}
\end{lemma}

\chapter{Steenrod homology}

\input{Steenrod_homology}



%A.4.9
%A.4.8
%A.4.20